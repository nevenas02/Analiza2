\documentclass[a4paper,12pt]{article}

\usepackage{type1ec}
\usepackage[OT2]{fontenc}
\usepackage[serbianc]{babel}
\usepackage{amsmath}
\usepackage{amsthm}
\usepackage{mathrsfs}
\usepackage{dsfont}
\usepackage{amsfonts}
\usepackage{mathtools}
\usepackage{empheq}
\usepackage{amssymb}
\usepackage{color} 
\usepackage{enumitem,framed} 
\usepackage{indentfirst}
\usepackage{hyperref} 
\usepackage{tikz}
\usepackage{nicefrac}
\usepackage{pgf}
\usepackage{pgfplots}
\usepackage[lmargin=2.0cm, rmargin=2.0cm,tmargin=2.50cm,bmargin=2.50cm]{geometry}

\newcommand{\latin}{\fontencoding{T1}\selectfont}
\newcommand{\nap}{\noindent\textbf{Napomena.} }
\newcommand{\f}{\mathscr{F}}
\newcommand{\N}{\mathbb{N}}
\newcommand{\Z}{\mathbb{Z}}
\newcommand{\Q}{\mathbb{Q}}
\newcommand{\R}{\mathbb{R}}
\newcommand{\NN}{\mathbb{N}}
\newcommand{\ZZ}{\mathbb{Z}}
\newcommand{\QQ}{\mathbb{Q}}
\newcommand{\RR}{\mathbb{R}}
\newcommand{\CC}{\mathbb{C}}
\newcommand{\eps}{\varepsilon}
\newcommand{\ps}{\subset}
\newcommand{\psj}{\subseteq}
\newcommand{\ds}{\displaystyle}
\newcommand{\norm}[1]{\left\lVert#1\right\rVert} 
\newcommand{\rang}{\mathrm{rang}}
\newcommand{\diam}{\mathrm{diam}}
%\renewcommand{\dim}{\mathrm{dim}}
\newcommand{\hdr}[1]{{\mathrm{H_{dR}}}^{#1}}
\newcommand{\const}{\mathrm{const}}
\newcommand{\grad}{\mathrm{grad}}
\newcommand{\rot}{\mathrm{rot}}
\renewcommand{\div}{\mathrm{div}}
\renewcommand{\vec}{\overrightarrow}


\newcommand{\tma}{\noindent\textbf{Teorema.} }
\newcommand{\tvr}{\noindent\textbf{Tvrd1enje.} }
\newcommand{\posl}{\noindent\textbf{Posledica.} }
\newcommand{\lema}{\noindent\textbf{Lema.} }
\newcommand{\pr}{\noindent\textit{Primer.} }
\renewcommand{\proofname}{\textbf{Dokaz.}}

\def\zn{,\kern-0.09em,} % definisemo donje navodnike 
\def\zng{'\kern-0.09em' } % definisemo gornje navodnike

\pagenumbering{gobble} 

\title{\textbf{Analiza 2 -- osnovne ideje dokaza vaz1nijih teorema i tvrd1enja }}
\date{}
\begin{document}

\maketitle

\section{Neprekidnost}

\begin{tvr}
(Koshi - Shvarcova nejednakost) Ako je $\langle \cdot, \cdot \rangle$ pozitivna hermit\-ska forma, onda je 
\[{|\langle x, y \rangle |}^2 = \langle x, x \rangle \langle y, y \rangle\]
\end{tvr}
\begin{proof}
Neka su $x, y \in X$ i $\alpha \in \RR$. Zbog pozitivnosti $0 \leq \langle x + \alpha y, x + \alpha y \rangle$. Raspisati prethodni izraz, zameniti $\alpha = \frac{- \langle x, y \rangle }{\langle y, y \rangle}$ i izvesti traz1eno. U sluchaju da je $\alpha \in \CC$ pokazati da je $|\langle x, y \rangle | = {( \mathrm{Re} \langle x, \alpha y \rangle)}^2$. Svesti na prvi sluchaj.
\end{proof}

\begin{tvr}
	Neka je $f:X\to Y$ i $g: Y\to Z$ i $f C a$, $g C f(a)$. Tada je $g\circ f C a$.
\end{tvr}
\begin{proof}
	Iskoristiti karakterizaciju neprekidnosti:
	\[ (\forall V \in \mathcal F_{f(a)})(\exists U \in \mathcal F_{a}) f(U)\psj V .\]
\end{proof}

\begin{tvr}
Za linearno preslikavanje $L:X \to Y$ normiranih vektorskih prostora sledec1i iskazi su ekvivalentni:
\begin{itemize}
\item[a)] $L$ je neprekidno u jednoj tachki
\item[b)] $L$ je neprekidno
\item[v)] $L$ je ravnomerno neprekidno
\item[g)] $L$ je Lipshicovo
\item[d)] $\sup_{\norm{x} = 1} \norm{L(x)} < \infty$
\end{itemize}
\end{tvr}
\begin{proof}
Primetiti da g) $\implies$ v)$ \implies$ b) $ \implies$ a) uvek vaz1i. Dalje, d)$ \implies$ g) sledi iz definicije $\mathrm{lip} f$. Potrebno je josh dokazati a) $\implies$ d). Kako iz $LCa$ sledi da je $L$ ogranicheno na nekoj lopti $B[a; \rho]$,tj. ogranicheno je i na $S[a; \rho]$. Predstaviti $S[a;\rho]$ kao $a + \rho S[0;1]$ i odale izvesti zakljuchak.
\end{proof}

\begin{nap}
	Ukoliko je $L$ vishelinearno preslikavanje, onda se na slichan nachin mozhe definisati $\norm L _{\infty}$. Tada
	vazhi: $L$ je neprekdino akko je $\norm L _{\infty} < \infty$. Primeri ovoga su mnozhenje, determinanta, 
	mnozhenje skalarom, vektorski proizvod, hermitski proizvod, meshoviti, slaganje neprekidnih
	linearnih preslikavanja, \dotso
\end{nap}

\newpage

\begin{tvr}
	Neka je $\ds{(X,d_\infty) = \prod_{j=1}^{\infty}(X_j, d_j)}$, $A_j\psj X_j$, $\ds{A = \prod_{j=1}^{\infty}A}$. Tada vazhi:
	\begin{enumerate}
		\item[(1)] Projekcije $\pi_j:X \to X_j$ su otvorena preslikavanja.
		\item[(2)] Unutrashnjost skupa $A$ je proizvod unutrashnjosti skupova $A_j$. Specijalno, $A$ je otvoren akko je otvoren \zn po koordinatama\zng.
		\item[(3)] Zatvorenje skupa $A$ je proizvod zatvorenja skupova $A_j$. Specijalno, $A$ je zatvoren akko je otvoren \zn po koordinatama\zng.
	\end{enumerate}
\end{tvr}
\begin{proof}
	Iskoristiti da je $\ds{B(a, \delta) = \prod_{j=1}^{\infty}B(a_j, \delta)}$
\end{proof}

\begin{tma}
Neka su $X$ i $Y$ metrichki prostori. Za $f:X \to Y$ sledec1i iskazi su ekviva\-lentni:
\begin{itemize}
\item[a)] $f$ je neprekidno
\item[b)] Za svaki otvoren skup $V \psj Y$ skup $f^{-1}(V) \psj X$ je otvoren
\item[v)] Za svaki zatvoren skup $G \psj Y$ skup $f^{-1}(G)$ je zatvoren
\item[g)] Za svaki podskup $A \psj X$ je $f(\overline{A}) \psj \overline{f(A)}$
\end{itemize}
\end{tma}

\begin{proof}
a) $\implies$ g) : Dokazati da za $a\in \overline{A}$ vazhi $f(a) \in \overline{f(A)}$. Iskoristiti neprekidnost u tachki $a$. \\ 
g) $\implies$ v) : Dokazati da je $f^{-1}(G)$ jednako $\overline{f^{-1}(G)}$. \\
v) $\implies$ b) : Iskoristiti zatvorenost $f^{-1}(Y \setminus V) = X \setminus f^{-1}(V)$. \\
b) $\implies$ a) : Dokazati da $f^{-1}(W) \in \mathcal{F}_a$ gde je $V \psj W$ i $V$ otvoren, a $a \in X$ i $W \in \mathcal{F}_{f(a)}$. 
\end{proof}

\begin{tvr}
(Princip produz1enja jednakosti) Neka su $X, Y$ metrichki prostori, $f,g : X \to Y$ neprekidna preslikavanja, $A \psj X$. Tada vaz1i:
\begin{itemize}
\item[(1)] Skup $S = \{x \in X \mid f(x) = g(x) \}$ je zatvoren
\item[(2)] $f|_A = g|_A \implies f|_{\overline{A}} = g|_{\overline{A}}$
\end{itemize}
\end{tvr}
\begin{proof}
\begin{itemize}
\item[(1)] $f(x) = g(x) \Leftrightarrow d(f(x), g(x)) = 0$. Tada $S = {(d(f,g))}^{-1}(\{ 0 \}$ zatvoren zbog neprekidnosti $ d(f,g)$.
\item[(2)] $f|_A = g|_A \implies A \psj S \implies \overline{A} \psj \overline{S} = S \implies f|_{\overline{A}} = g|_{\overline{A}}$
\end{itemize}
\end{proof}

\begin{tvr}
(Princip produz1enja nejednakosti) Neka je $X$ metrichki prostor, $A \psj X$ i $f,g : X \to \RR$ neprekidna preslikavanja. Tada vaz1i:
\begin{itemize}
\item[1)] Skup $V = \{x \in X \mid f(x) < g(x) \}$ je otvoren
\item[2)] Skup $V = \{x \in X \mid f(x) \leq g(x) \}$ je zatvoren
\item[3)] $f|_A \leq g|_A \implies f|_{\overline{A}} \leq g|_{\overline{A}}$
\item[4)] $\sup f(A) = \sup f(\overline{A})$ (analogno i za $\inf$)
\end{itemize}
\end{tvr}

\begin{proof}
Definisati preslikavanje $\varphi := g - f : X \to \RR$ i za $1)$ i $2)$ posmatrati inverzne slike skupova $[0, \infty)$ i $]0, \infty)$. Pod $3)$ se dokazuje kao u prethodnom tvrd1enju. $4)$ sledi iz $3)$ ako uzmemo da je $g = C$ gde je $C$ konstanta kojom je ogranichena restrikcija $f$ na $A$.
\end{proof}

\begin{nap}
U prethodnom tvrd1enju kodomen funkcija $f$ i $g$ je $\RR$ zbog toga shto nam je zbog nejednakosti potrebno ured1enje na kodomenu.
\end{nap}
\\ \\
\begin{posl}
	$\diam(A) = \diam(\overline{A})$, $d(A, B) = d(\overline A, \overline B)$
\end{posl}
\begin{proof}
Sledi iz principa produz1enja nejednakosti.
\end{proof}

\begin{tvr}
(Smena promenljive u limesu) Neka su $X, Y$ metrichki prostori, $A \psj X$, $a \in \overline{A}$. Tachka $\xi \in Y$ je granichna vrednost preslikavanja $f$ u tachki $a$. Neka je $\varphi: M \to X$ homeomorfizam metrichkog prostora $M$ na $X$. Tada je 
\[\lim_{x \to a} f(x) = \lim_{t \to {\varphi}^{-1}(a)} f \circ \varphi(t)\]
\end{tvr}
\begin{proof}
Formirajmo niz $t_n \in \varphi^{-1} (A)$ koji tez1i ka $\varphi^{-1}(a)$. Tada $f(\varphi(t_n)) \to \xi$ zbog nepre\-kidnosti $\varphi$ u $a$ i zbog $\lim_{x \to a} f(x) = \xi$.
\end{proof}

\begin{tvr}(Princip produzhenja nejednakosti)
	Neka je $A\psj X, a\in \overline A$, $f,g:A\to \RR$, \( \lim_{x\to a}f(x) = \alpha \) , \( \lim_{x\to a} = \beta \). Tada vazhi:
	\begin{enumerate}
		\item[(1)] $(\exists B)(B\psj A \land a \in \overline B \land f|_B \leq g|_B) \implies \alpha \leq \beta$
		\item[(2)] $\alpha < \beta \implies (\exists V \in \mathcal F _a^{0})f|_{A \cap V} \leq g|_{A\cap V}$
	\end{enumerate}
\end{tvr}
\begin{proof}
	\begin{enumerate}
		\item[(1)] Formirati niz $B \ni b_n \to a$ i primeniti saglasnost limesa sa $\leq$.
		\item[(2)] Postoji $\gamma, \alpha < \gamma < \beta$. Kako su $(-\infty, \gamma)$ i $(\gamma, +\infty)$ okoline tachaka $\alpha$ i $\beta$, redom,
			ti intervali su ujedno i elementi filtera (probushenih) okolina tachaka $\alpha$ i $\beta$, redom. Iz chinjenice da je filter zatvoren u odnosu
			na presek, izvesti zakljuchak.
	\end{enumerate}

\end{proof}

\subsection{Kompletnost}

\begin{tvr}
	Ravnomerno neprekidna slika Koshijevog niza je Koshijev niz.
\end{tvr}
\begin{proof}
	Primeniti definiciju.
\end{proof}

\begin{tvr}
Podskup $A \psj X$ kompletnog prostora $X$ je kompletan akko je zatvoren.
\end{tvr}
\begin{proof}
$\boxed{\Rightarrow}$ Dokazati da je $A = \overline{A}$ preko konvergentnog niza.\\
$\boxed{\Leftarrow}$ Iz zatvorenosti svaki Koshijev niz je konvergentan u $A$.
\end{proof}

\begin{nap}
Primeri prostora koji nisu kompletni su nezatvoreni podskupovi komple\-tnih.
\end{nap}

\begin{tvr}
Proizvod $(X, d_{\infty}) = \prod_{j=1}^{k} (X_i, d_i)$ je kompletan ako i samo ako su svi $X_i$ kompletni.
\end{tvr}
\begin{proof}
Neka je $\pi_j : X \to X_j$. Dokazati da nizovi $\pi_j a_n$ konvergiraju u prostorima $X_j$, gde je $a_n$ niz u $X$. Za drugi smer, utopiti $X_j$ u $X$
	kao $i:X_j \to X_j\times \{c_1\} \times \{c_2\} \dotso$
\end{proof}

\begin{tma}
(Kantorova karakterizacija kompletnosti) Metrichki prostor $(X,d)$ je kom-pletan ako i samo ako svaki opadajuc1i niz nepraznih zatvorenih podskupova chiji niz dijametara tezi nuli ima neprazan presek.
\end{tma}
\begin{proof}
$\boxed{\Rightarrow}$ Pretpostavimo da je prostor kompletan. Neka je $F_n$ niz podskupova iz uslova teoreme. Formiramo niz $a_n \in F_n$. Pokazi da je taj niz Koshijev, pa je zbog kompletnosti i konvergentan, stoga presek je $a_{\infty}$ pa nije prazan. \\
$\boxed{\Leftarrow}$ Neka je $a_n$ Koshijev niz u $X$. Definisati $F_n := \overline{\{a_n ,a_{n+1}, ...\}}$, pa kako $X$ ima Kantorovo svojstvo, presek nije prazan, tj postoji $a_{\infty} \in \bigcap_{n \in \NN} F_n$.
\end{proof}

\begin{tma}
(Produz1enje neprekidnog preslikavanja) Neka su $X, Y$ metrichki prostori, $A \in X$, $\overline{A} = X$ (tj. $A$ gust u $X$) i $f: A \to Y$ neprekidno preslikavanje. Tada vaz1i:
\begin{itemize}
\item[(1)] Postoji neprekidno preslikavanje $F: X \to Y$ sa osobinom $F|_{A} = f$ ako i samo ako $(\forall x \in X)$ $\exists \lim_{t \to x} f(t) =: g(x)$. Pri tome je $F \equiv g$ (jedinstvenost produz1enja).
\item[(2)] Ako je $f$ ravnomerno neprekidno i $Y$ kompletan, onda postoji ravnomerno nepre\-kidno $F: X \to Y$ sa osobinom $F|_{A} = f$.
\end{itemize}
\end{tma}
\begin{proof}
\begin{itemize}
	\item[(1)] Jedan smer je trivijalan. Za drugi smer iskoristiti svojstvo neprekidne funkcije $f(\overline{A}) \psj \overline{f(A)}$,
		primenjeno na odgovarajuc1u okolinu tachke.
\item[(2)] Neka je $x\in X$ proizvoljna tachka i $a_n$ niz u $A$ koji tezhi ka $x$. Tada je $a_n$ Koshijev, pa 
	je $f(a_n)$ takodje Koshijev (jer je $Y$ kompletan i $f$ ravnomerno neprekidno), pa je $f(a_n)$ konvergentan.
		Da bismo mogli da kazhemo $F(x):= \lim_{n\to\infty}f(a_n)$, moramo dokazati da ova definicija ne zavisi od izbora
		niza. Pretpostavimo da $b_n\to x$. Dokazati tada $d_X(a_n, b_n)\to 0$ i odatle, iz ravnomerne neprekidnosti, zakljuchiti
		da $d_Y(f(a_n), f(b_n))\to 0$, odakle sledi dobra definicija (ovo je smisleno zbog kara-kterizacije limesa u tachki preko 
		limesa niza).
\end{itemize}

\end{proof}

\begin{tma}
(Banahova teorema o fiksnoj tachki) Neka je $X$ kompletan metrichki prostor, $f: X \to X$ kontrakcija, $0 < q = \mathrm{lip} f < 1$. Tada $f$ ima tachno jednu fiksnu tachku, tj. postoji jedinstvena tachka $a_{\infty} \in X$ takva da je $f(a_{\infty}) = a_{\infty}$.\\ \\
Tachka $a_{\infty}$ je granichna vrednost niza definisanog rekurentno: $a_0$ proizvoljno, $a_{n+1} = f(a_n)$ (niz sukcesivnih aproksimacija) i pri tome vaz1i \[d(a_n, a_{\infty}) \leq \frac{q^n}{1-q} d(a_0, a_1)\]
\end{tma}

\begin{proof}
Pocheti od $d(a_{n+2}, a_{n+1}) = d(f(a_{n+1}), f(a_n)) \leq q d(a_{n+1}, a_n)$. Pomnoz1iti nejednakost sa $q^{-(n+1)}$. Zakljuchiti da je niz $q^{-n} d(a_{n+1}, a_n)$ opadajuc1i. Proceniti $d(a_{n+1}, a_n)$ sa $d(a_1, a_0)$. Proceniti $d(a_m, a_n)$ (supremum toga) i zakljuchiti da je $a_n$ Koshijev. Tad je fiksna tachka limes tog niza. Dokazati jedinstvenost suprotnom pretpostavkom.
\end{proof}

\begin{tma}
Normirani vektorski prostor $X$ je kompletan ako i samo ako je svaki apsolu\-tno konvergentan red u njemu konvergentan.
\end{tma}
\begin{proof}
$\boxed{\Rightarrow}$ Neka je $X$ kompletan i $\sum_{n=0}^{\infty} \norm{a_k}<\infty$. Neka je $s_n = \sum_{k=0}^n a_k$. Dokazati da je $s_n$ Koshijev. \\
$\boxed{\Leftarrow}$ Neka je $a_n$ Koshijev u $X$. Tada postoji $\varphi: \NN \nearrow \NN$ td. $\norm{a_{\varphi(n)} - a_{\varphi(n+1)}} \leq 2^{-n}$.
	Takvo $\varphi$ konstruishemo na sledec1i nachin
	\begin{align*}
		A_n &= \{ a_n, a_{n+1},\dotso \} \\
		\varphi(0) :&= \min\{n \in \NN \mid \mathrm{diam}(A_n) < 2^0 \}\\
		\varphi(n+1) :&=\min\{k \in \NN \mid k>\varphi(n) \land \mathrm{diam}(A_k) < 2^{-(n+1)} \} 
	\end{align*}
	Poshto $\sum 2^{-n}$ konvergira, zakljuchiti i da $a_{\varphi(n)}$ konvergira. Iskoristiti chinjenicu da je Koshijev niz koji ima konvergentan podniz i sam konvergentan. Sledi kompletnost $X$.
\end{proof}

\begin{tma}
$B_Y(X)$ je Banahov ako i samo ako je $Y$ Banahov.
\end{tma}
\begin{proof}
	$\boxed{\Rightarrow}$ Neka je $B_Y(X)$ kompletan i $b_n$ niz u $Y$. Hoc1emo da iskoristimo prethodnu teoremu, tj. pretpostavimo da red sa 
	opshtim chlanom $b_n$ apsolutno konvergira. Defini\-shimo $f_n:X\to Y$, $f_n(x) \equiv b_n$. Zakljuchiti da red sa opshtim chlanom $f_n$ konvergira
	(u $B_Y(X)$), pa odatle izvesti zakljuchak da $\sum b_n$ konvergira. \\
	$\boxed{\Leftarrow}$ Slichna ideja kao u direktnom smeru. Iskoristiti josh i poredbeni princip.
\end{proof}

\begin{posl}
$\mathcal{L}(X;Y)$ je kompletan ako i samo ako je $Y$ kompletan.
\end{posl}

\begin{proof}
Dokaz sledi iz prethodnog i toga da je $\mathcal{L}(X;Y)\hookrightarrow B_Y(S)$ izomorfizam na zatvoren podskup,  gde je $S = \{x \in X \mid \norm{x} = 1 \}$.
\end{proof}

\begin{nap} Ako je $X$ normirani vektorski prostor nad poljem $\mathbb{K} \in \{\RR, \CC \}$, onda je njegov dual $X^* := \mathcal{L} (X; \mathbb{K})$ kompletan.
\end{nap}
\\ \\
\begin{nap}
U konachno dimenzionom sluchaju $X \cong X^*$, tj vaz1i Risova teorema o re\-pre\-zen\-ta\-ci\-ji
\[(\forall \alpha \in X^*) (\exists a \in X) (\forall v \in X) \hspace{0.12cm} \alpha(v) = \langle v, a \rangle \]
\end{nap}
\\ \\
\begin{nap}
U beskonachno dimenzionom slucaju ne mora da vaz1i prethodno, $X$ ne mora biti kompletan, a $X^*$ je uvek kompletan.
\end{nap}
\\ \\
\begin{nap}
Ako je $X$ Hilbertov, onda je $X \cong X^*$.
\end{nap}

\subsection{Kompaktnost}

\begin{tvr}
Kompaktan metrichki prostor je kompletan.
\end{tvr}
\begin{proof}
Sledi iz chinjenice da ako Koshijev niz ima konvergentan podniz, onda je i sam konvergentan.
\end{proof}

\begin{tvr}
\begin{itemize}
\item[(1)] Kompaktan skup je zatvoren
\item[(2)] Zatvoren podskup kompaktnog skupa je kompaktan
\end{itemize}
\end{tvr}
\begin{proof}
\begin{itemize}
\item[(1)] Ranije dokazano za kompletnost
\item[(2)] Trivijalno
\end{itemize}
\end{proof}

\begin{tvr}
Neprekidna slika kompaktnog skupa je kompaktan.
\end{tvr}
\begin{proof}
Konstruisati niz $b_n$ u $f(X)$. Nac1i njegov konvergentan podniz.
\end{proof}

\begin{posl}
Neprekidna realna funkcija na kompaktnom skupu je ogranichena i ima $\min$ i $\max$.
\end{posl}
\\
\begin{tvr}
Proizvod metrichkih prostora je kompaktan ako i samo ako su takvi prostori faktori.
\end{tvr}
\begin{proof}
$\boxed{\Rightarrow}$ projekcije su neprekidne surjekcije pa su ima slike kompaktne \\
$\boxed{\Leftarrow}$ princip dijagonalizacije
\end{proof}

\begin{tvr}
Kompaktan skup je ogranichen.
\end{tvr}
\begin{proof}
Neprekidnost $f: A \to \RR$ $f(x)=d(x,a)$
\end{proof}

\begin{tvr}
Neka su $X$ i $Y$ metrichki prostori, $X$ kompaktan, $f: X \to Y$ neprekidna bijekcija. Tada je $f$ homeomorfizam, tj. preslikavanje $f^{-1}:Y \to X$ je takod1e neprekidno.
\end{tvr}
\begin{proof}
Neka je $F \psj X$ zatvoren podskup. Tada je $F$ kompaktan, pa je i $f(F)$ kompaktan, odnosno $f(F)$ je zatvoren. Zakljuchak sledi iz chinjenice da je neprekidna slika zatvorenog skupa zatvorena akko je preslikavanje neprekidno.
\end{proof}

\begin{nap}
Preslikavanje $\varphi: [0,2 \pi[ \to \mathbb{S}^1$ dato kao $ \varphi(t):= e^{it}$ je neprekidna bijekcija koja nije homeomorfizam.
\end{nap}

\begin{tma}
(Kantorova) Neprekidno preslikavanje na kompaktu je ravnomerno neprekidno.
\end{tma}
\begin{proof}
Pretpostaviti suprotno, da je $X$ kompaktan, $f:X \to Y$ neprekidno, ali nije ravnomerno. Negirati uslov ravnomerne neprekidnosti i uzeti $\delta = \frac{1}{n}$.  Izvesti kontradikciju. 
\end{proof}

\begin{lema}
Svaka norma 
\[\norm{\cdot} : \mathbb{K}^r \to \RR \]
na vektorskom prostoru $\mathbb{K}^r$ je Lipshicovo preslikavanje normiranog vektorskog prostora $(\mathbb{K}^r, \norm{\cdot})$ sa standardnom euklidskom normom.
\end{lema}
\begin{proof}
Predstaviti $x = (x_1, \dots, x_r)$ kao linearnu kombinaciju baznih vektora. Primenom Koshi-Shvarcove nejednakosti zakljuchiti da je $\norm{x} \leq C |x|$, odnosno dalje, $|\norm{x} - \norm{y} | \leq \norm{x-y} \leq C |x-y|$.
\end{proof}

\begin{tvr}
Neka je $\norm{\cdot}$ proizvoljna norma. Tada 
\[(\exists m,M \in (0,\infty) ) (\forall x \in \mathbb{K}^r) \hspace{0.1cm} m|x| \leq \norm{x} \leq M|x| \]
gde je $|\cdot|$ standardna euklidska norma na $\mathbb{K}^r$.
\end{tvr}
\begin{proof}
Poshto je jedinichna sfera kompaktan skup, $\norm{\cdot}$ dostiz1e minimum i maksimum na njemu. Vaz1i
 \[ m \leq \norm{\frac{x}{|x|}} \leq M\]
\end{proof}

\begin{nap}
Na $k-\dim$ prostoru sve norme su ekvivalentne.
\end{nap}
\\ \\
\begin{posl}
Svaka dva normirana vektorska prostora iste konachne dimenzije su algebarski i topoloshki izomorfna.
\end{posl}
\\ \\
\begin{nap}
Dimenzija je jedina invarijanta konachno dimenzionog vektorskog prostora.
\end{nap}
\\ \\
\begin{tvr}
Metrichki prostor je totalno ogranichen (pred-kompaktan) ako i samo ako svaki niz u njemu ima Koshijev podniz.
\end{tvr}
\begin{proof}
$\boxed{\Rightarrow}$ Neka je $X$ totalno ogranichen i neka je $a \in X^{\NN}$. Neka je $E_1$ 1-mrez1a. Tada se za neko $e_1 \in E_1$ u lopti $B_1 = B]e1;1[$ nalazi beskonachno mnogo chlanova niza $a_n$. Dalje, ponavljati postupak za $\frac{1}{n}$ - mrez1e. Dobija se niz lopti $B_n$ takvih da je svaka sledec1a sadrz1ana u prethodnoj. Trazeni niz definisati kao $a \circ \varphi$ gde je $\varphi: \NN \nearrow \NN$ dato sa: \[\varphi(0) = 0, \varphi(n+1) = \min a^{-1} (B_n \cap (\varphi(n), + \infty))\]
$\boxed{\Leftarrow}$ Pretpostaviti suprotno, da $X$ nije totalno ogranichen. Izvesti kontradikciju.
\end{proof}

\begin{posl}
Metrichki prostor je kompaktan ako i samo ako je kompletan i totalno ogranichen.
\end{posl}

\begin{tvr}
Podskup $A \psj X$ kompletnog metrichkog prostora $X$ je relativno kompaktan, odnosno njegovo zatvorenje je kompaktno, ako i samo ako je totalno ogranichen.
\end{tvr}
\begin{proof}
Iz kompaktnosti $\overline{A}$ zakljuchiti da je $\overline{A}$ kompletan i totalno ogranichen. Primenom produz1enja nejednakosti zakljuchiti da je i $A$ totalno ogranichen.
\end{proof}

\begin{nap}
Svaki kompaktan podskup u Hausdorfovom prostoru je relativno kompaktan
\end{nap}
\\ \\
\begin{lema}(Lebegova)
Neka je $X$ kompaktan metrichki prostor i $(V_{\lambda})_{\lambda \in \Lambda}$ njegovo otvoreno pokrivanje. Tada $(\exists \delta > 0)$ sa svojstvom da je svaka otvorena lopta poluprechnika $\delta$ sadrz1ana u nekom $V_{\lambda}$.
\end{lema}

\subsection{Prostor neprekidnih preslikavanja}

\begin{tvr}
$X$ kompaktan $\implies$ $C(X;Y) \psj B(X;Y)$
\end{tvr}
\begin{proof}
Zakljuchiti iz toga da funkcija $f:X \to \RR$ koja slika $x\mapsto d(b,f(x))$ na kompaktu dostiz1e minimum i maksimum.
\end{proof}

\begin{tvr}
$C_Y(X)$ je zatvoren u $B_Y(X)$.
\end{tvr}
\begin{proof}
Neka je $f_n:X \to Y$ niz u $C_Y(X)$. Neka $f_n \longrightarrow f_{\infty}$ kad $n \to \infty$. Dokazati da je $f_{\infty} \in C_Y(X)$, odnosno da je $(\forall
a \in X) fCa$. Iskoristiti definiciju neprekidnosti u tachki.
\end{proof}

\begin{posl}
Ako je $X$ kompaktan i $Y$ kompletan, onda je i $C_Y(X)$ kompletan.
\end{posl}
\begin{proof}
Zbog toga shto, ako je $X$ kompaktan tada je $C(X;Y) \psj B(X;Y)$. Ako je $Y$ kompletan, tada je $B(X;Y)$ kompletan. Prema prethodnom $C(X;Y)$ je zatvoren u $B(X;Y)$, pa stoga zakljuchujemo da je i kompletan kao zatvoren podskup kompletnog skupa.
\end{proof}

\begin{tma}
(Dini) Neka je $X$ kompaktan metrichki prostor i $f_n$ monoton niz u $C_{\RR}(X)$ takav da vaz1i:
\begin{itemize}
\item[(1)] $(\forall x \in X) \lim_{n \to \infty} f_n(x) = f_{\infty}(x)$
\item[(2)] $f_{\infty} \in C_{\RR}(X)$
\end{itemize}
Tada $f_n \rightrightarrows f_{\infty}$ (tj. $f_n \to F$ u $(C_{\RR}(X), d_{\infty})$.
\end{tma}
\begin{proof}
Na kodomenu $\RR$ metrika je definisana uz pomoc1 $|\cdot |$, dalje na $C_{\RR}(X)$ imamo normu $\norm{g}_{\infty}:= \max_{x\in X}|g(x)|$ i ravnomerna metrika je $d_{\infty}(f,g) = \norm{f - g}_{\infty}$. Posmatrajmo niz funkcija: $g_n:X \to \RR$, $g_n(x) = |f_{\infty}(x) - f_n(x)|$. Dokaz sprovesti u dva koraka:
\begin{itemize}
\item[1)] Iz osobina niza $g_n$ zakljuchiti da $\norm{g_n}_{\infty} \searrow$.
\item[2)] Formirati niz $F_n = \{x \in X | g_n(x) \geq \epsilon \}$. On je zatvoren. Dokazati da vaz1i \\$\lim_{n \to \infty} \norm{g_n}_{\infty} = 0$. Odatle sledi traz1eno.
\end{itemize}

Moguc1e je i posmatrati niz otvorenih skupova $U_n = \{x \in X | g_n(x) < \epsilon \}$. Ovo je otvoreni pokrivach kompaktnog prostora $X$, odakle
	sledi da imamo konachan potpokrivach. Kako je $g_n$ opadajuc1i niz, odabirom maksimalnog $n$ iz konachnog potpokrivacha sledi da je $\norm{g_n}_\infty < \eps$,
	a kako je $\eps$ proizvoljno, odatle sledi dokaz.
\end{proof}

\begin{tma}
(Korovkinova) Neka je $X$ kompaktan metrichki prostor i $S \psj C_{\RR}(X)$ podskup, takav da je njime generisani vektorski prostor minimizirajuc1i. Ako je \[P_n: C_{\RR}(X) \to C_{\RR}(X)\]
 niz pozitivnih linearnih preslikavanja takav da $(\forall f \in S) P_n f \rightrightarrows f$, tada vaz1i
\[(\forall f \in C_{\RR}(T)) P_nf \rightrightarrows f \]
\end{tma}

\begin{proof}
Zbog linearnosti uniformne konvergencije i linearnosti $P_n$ moz1emo da pretpo\-stavimo da je $S = M$.  Neka je $f \in C_{\RR}(X)$. Izvesti dokaz u tri koraka:
\begin{itemize}
\item[1)] Neka je $\epsilon >0$. Tada 
\[(\forall a \in X)(\exists g \in M) g(a) \leq f(a) + \epsilon \wedge g > f\]
\item[2)] $(\forall \epsilon > 0)$ $\exists$ konachan niz $g_1, \dots, g_r \in M$ takav da je $f < \inf \{g_1, \dots, g_r \} \leq f + \epsilon $
\item[3)] Primenom prethodnog, dokazati da je $\norm{P_n f - f}_{\infty} = \sup_{x \in X} |P_nf - f|<\epsilon$.
\end{itemize}
\end{proof}

\begin{pr}
(Bernshtajnovi polinomi)
$B_n: C[0,1] \to C[0,1]$
\[B_n f(x) = \sum_{j=0}^n {{n}\choose{j}} f\bigg(\frac{j}{n}\bigg) x^j {(1-x)}^{n-j}\]
Neka je $S = \{e_0, e_1, e_2 \}$, $e_0(x) = 1, e_1(x) = x, e_2(x) = x^2$ i neka je $M$ vektorski prostor\\ sa bazom $\{e_0,e_1,e_2\}$. $M \psj C[0,1]$, $M$ je minimizirajuc1i. Vaz1i da $B_n e_j \rightrightarrows e_j$ za sve \\$j \in \{0,1,2\}$. Po Korovkinovoj teoremi zakljuchujemo: \[(\forall f \in C[0,1])\hspace{0.15cm} B_n f \rightrightarrows f \].
\end{pr}
\\ \\
\begin{posl}
(Prva Vajershtrasova teorema) Skup polinoma je gust u $(C[a,b], \norm{\cdot}_{\infty})$.
\end{posl}
\begin{proof}
Homeomorfizam $\varphi:[0,1] \to [a,b]$ gde je $\varphi(t) = bt + a(1-t)$ definishe linearnu izmometriju $\varphi^*: C[a,b] \to C[0,1]$ sa $\varphi^* f = f \circ \varphi$. Dalje razmatranje svodi se na prethodni primer.
\end{proof}

\begin{nap}
Profinjenje Prve Vajershtrasove teoreme je razvoj u stepeni red.
\end{nap}
\\ \\
\begin{tma}
(Ston - Vajershtrasov stav)
\begin{itemize}
\item Realni oblik: Ako je $\mathcal{A}$ podalgebra algebre $C_{\RR}(X)$ koja sadrz1i $1$ i razlikuje tachke, tj. vaz1i $(\forall x,y \in X) x \ne y \implies (\exists f \in \mathcal{A}) f(x) \ne f(y)$, onda je $\mathcal{A}$ gusta u $C_{\RR}(X)$: $\overline{\mathcal{A}} = C_{\RR}(X)$.
\item Kompleksni oblik: Ako je $\mathcal{A}$ podalgebra algebre $C_{\CC}(X)$ koja sadrz1i $1$ i razlikuje tachke i zatvorena je za kon\-jugovanje, tj. $f \in \mathcal{A} \implies \overline{f} \in \mathcal{A}$, onda je $\mathcal{A}$ gusta u $C_{\CC}(X)$.
\end{itemize}
\end{tma}
\begin{proof}

\end{proof}

\begin{posl}
(Druga Vajershtrasova teorema)
\begin{itemize}
\item Algebra generisana skupom $\{e^{i n t}\}$ je gusta u $C_{\CC}(2 \pi)$.
\item  Algebra generisana skupom ${\{1, \cos nt, \sin nt \}}_{n \in \NN}$ je gusta u $C_{\RR}(2 \pi)$.
\end{itemize}
\end{posl}

\begin{nap}
Profinjenje Druge Vajershtrasove teoreme jeste razvoj u Furijeov red.
\end{nap}
\\ \\
\begin{tma}
(Arcela - Askolijeva teorema) Neka je $X$ separabilan i $f_n: X \to \CC$ niz koji je:
\begin{itemize}
\item[1.] ogranichen po tachkama, tj. $(\forall x \in M)$ $\sup_{n \in \NN} |f_n(x)| < \infty$
\item[2.] ravnostepeno neprekidan
\end{itemize}
Tada niz $f_n$ ima podniz koji ravnomerno konvergira na svakom kompaktnom podskupu.
\end{tma}
\begin{proof}
Poshto je $X$ separabilan, postoji prebrojiv gust podskup $S \ps X$. Pored1ajmo njegove tachke u niz $S = \{x_1, x_2, \dots \}$. Definishimo niz podskupova $N_k \ps \NN$ induktivno. Neka je $N_0 = N$. Neka je $N_k$ definisano. Prema uslovu $1.$ postoji beskonachan skup $N_{k+1} \ps N_k$, takav da postoji 
\[\lim_{ N_{k+1}\ni n \to \infty} f_n(x_{k+1}) \]
Formirajmo $N := \{n_1, n_2, \dots \}$ gde je $n_k$ $k$ - ti element iz skupa $N_k$ za svako $k$ - dijagonalni izbor. Iz konstrukcije
sledi da za svako $x\in S$ postoji
	\[ \lim_{N\ni n \to \infty}f_n(x) .\] 
	Neka je $K\ps M$ kompaktan podskup. Pokriti $K$ sa konachno mnogo otvorenih lopti, koje su dovoljno malog poluprechnika takvog
	da je za svako  $n$ vrednost $|f_n(x) - f_n(y)|$ proizvoljno mala (ovo je moguc1e zbog ravnostepene neprekidnosti).
	Iskoristiti sada i da postoji tachka u preseku $S$ i svake takve lopte. Na kraju dovrshiti dokaz ocenom
	$|f_n(x) - f_m(x)|$ uz pomoc1u nejednakosti trougla, Kosijevog uslova (u kompletnom prostoru $\CC$).
\end{proof}

\begin{nap}
U Arcela - Askolijevoj teoremi $X$ moz1e biti i kompaktan, poshto je kompakt\-nost jache svojstvo od separabilnosti.
\end{nap}
\subsection{Povezanost}

\begin{tma}
Za topoloshki prostor $(X, \tau)$ sledec1i iskazi su ekvivalentni:
\begin{itemize}
\item[a)] $X$ ne moz1e da se predstavi kao unija $2$ disjunktna neprazna otvorena skupa
\item[b)]  $X$ ne moz1e da se predstavi kao unija $2$ disjunktna neprazna zatvorena skupa
\item[v)] Jedini podskupovi u $X$ koji su i otvoreni i zatvoreni su $\emptyset$ i $X$
\item[g)] Svako neprekidno preslikavanje $X \to \{0,1\}$ (gde je $\{0,1\}$ metrichki prostor sa diskretnom metrikom $d(0,1) = 1, d(0,0) = d(1,1) = 0$) je konstantno
\end{itemize}
\end{tma}

\begin{proof}
a) $\Leftrightarrow$ b) : Pretpostaviti suprotno. Zakljuchak sledi iz chinjenice da ako je $A$ otvoren tada je $B = X \setminus A$ zatvoren i obrnuto. \\ 
a) $\implies$ v) : Pretpostaviti suprotno, da postoji $A \subset X$ koji je i otvoren i zatvoren i neprazan. Tada se $X$ moz1e predstaviti kao unija dva otvorena disjunktna skupa $A$ i $X \setminus A$ odakle sledi kontradikcija sem ako $A$ nije prazan.
\\ 
v) $\implies$ a) : Slichno prethodnom.
\\
a) $\implies$ g) : Pretpostaviti suprotno, da $f$ nije konstantno. Dokazati da tada ne vaz1i a). Kontrapozicija.
\\
g) $\implies$ a) : Pretpostaviti $\neg$ a). Koristec1i chinjenicu da je inverzna slika otvorenog skupa otvoren skup zakljuchiti $\neg$ g).
\end{proof}

\begin{tvr}
Ako je $S$ povezan skup i $S \psj A \psj \overline{S}$, onda je i $A$ povezan.
\end{tvr}
\begin{proof}
Posmatrati neprekidno preslikavanje $f: A \to \{0, 1\}$ koje zbog povezanosti mora biti $\const$. Zakljchak sledi iz principa produz1enja jednakosti.
\end{proof}

\begin{tvr}
Ako je $\{S_{\lambda}\}_{\lambda \in \Lambda}$ familija povezanih skupova i $\bigcap_{\lambda \in \Lambda} S_{\lambda} \ne \emptyset$, onda je $\bigcup_{\lambda \in \Lambda} S_{\lambda}$ povezan skup.
\end{tvr}
\begin{proof}
Konstruisati neprekidno preskikavanje $f: \bigcup_{\lambda \in \Lambda} S_{\lambda} \to \{ 0, 1 \}$. Na svakom $S_\lambda$ ono je $\const$. Posto je presek njih neprazan, mora biti $\const$ i na celoj uniji.
\end{proof}

\begin{nap}
Vaz1i i uopshtenje: $\{S_{\lambda}\}_{\lambda \in \Lambda}$ povezani i $(\exists \lambda_0) (\forall \lambda \in \Lambda) S_{\lambda} \bigcap S_{\lambda_0} \ne \emptyset \implies \bigcup_{\lambda \in \Lambda} S_{\lambda}$ povezan - (teorema "cvetic1").
\end{nap}

\begin{posl}
Dekartov proizvod povezanih metrichkih prostora je povezan.
\end{posl}
\begin{proof}
Predstaviti $M \times N = \bigcup_{a\in M} \{a\} \times N = (\bigcup_{a\in M} \{a\} \times N) \cup
(M \times  \{y_0\})$. Dalje, sledi na osnovu prethodnog.
\end{proof}

\begin{tvr}
Neprekidna slika povezanog skupa je povezan skup.
\end{tvr}
\begin{proof}
Sledi iz definicije i surjektivnosti $f: X \to f(S)$.
\end{proof}

\begin{tvr}
Putno povezan skup je povezan.
\end{tvr}
\begin{proof}
Pokazati da je neprekidno $\varphi: S \to \{0,1\}$ konstantno. Zakljuchiti da je $\varphi \circ \gamma = \const$ gde je $\gamma:[0,1] \to S$ neprekidno takvo da je $\gamma(0) = s_0, \gamma(1) = x$. Odatle sledi $\varphi(x) = \varphi(s_0)$, tj. $\varphi = \const$.
\end{proof}

\begin{tvr}
Neprekidna slika putno povezanog skupa je putno povezan.
\end{tvr}
\begin{proof}
Konstruisati $f \circ \gamma$. Posmatrati shta je to preslikavanje u $0$ i $1$.
\end{proof}

\begin{nap}
Prethodni dokaz i dokaz slichnog tvrd1enja za obichnu povezanost su na neki nachin dualni ("metod obrnutih strelica").
\end{nap}

\section{Diferenciranje}
\subsection{Diferenciranje u normiranim vektorskim prostorima}

\begin{tma}(Izvod sloz1ene funkcije)
Neka su $X, Y,$ i $Z$ normirani vektorski prostori, $V \psj X$, $W \psj Y$ otvoreni skupovi, $f: V \to Y$, $g: W \to Z$ i $a \in V$, $f(a) \in W$. Tada vaz1i:
\[f\mathcal{D} a \wedge g \mathcal{D} f(a) \implies g\circ f \mathcal{D} a\]
i pri tome je
\[D(g \circ f) (a) = Dg(f(a))\cdot Df(a) \]
\end{tma}
\begin{proof}
Dokaz sledi iz definicije izvoda preslikavanja i chinjenice da je $o(O(h)) = o(h)$.
\end{proof}

\begin{tma}(Izvod inverzne funkcije)
Neka su $X$ i $Y$ normirani vektorski prostori, $V \psj X$ otvoren skup, $a\in V$ i $f: V \to Y$ preslikavanje sa sledec1im svojstvima:
\begin{itemize}
\item[1)] $f \mathcal{D} a$
\item[2)] $Df(a)$ invertibilan
\item[3)] u nekoj $W$ okolini tachke $b := f(a)$ preslikavanje $f$ ima inverzno
\item[4)] $f^{-1} C b$
\end{itemize}
Tada je $f^{-1} \mathcal{D} b$ i vaz1i:
\[(Df^{-1})(b) = {(Df(a))}^{-1}\]
\end{tma}
\begin{proof}
Neka je $b = f(a), a = f^{-1}(b)$. Pokazati da, poshto su translacije homeomorfizmi, vaz1i $f(a+h) = b+t$ i $f^{-1}(b+t) = a+h$. Iz neprekidnosti $f^{-1}$ u $b$ zakljuchiti chemu je jednako $f^{-1} (b+t) - f^{-1}(b)$. Iz uslova $f \mathcal{D} a$ zakljuchiti chemu je jednako ${(f'(a))}^{-1} t$ kao i da je $o(h) = o(t)$. Na osnovu prethodnih koraka, i definicije izvoda, izvesti konachni zakljuchak.
\end{proof}

\begin{tvr}
Ako je prostor Banahov (kompletan normiran vektorski prostor), skup $GL(X)$ je otvoren u $\mathcal{L}(X;X)$.
\end{tvr}
\begin{proof}
Posmatrati $A \in GL(X)$ i $h \in \mathcal{L} (X;X)$. Dokazati da je i $A+h \in GL(X)$ odnosno da postoji ${(A+h)}^{-1}$. Iskoristiti kompletnost domena kao i definisanost norme.
\end{proof}

\begin{nap}
Prethodno tvrd1enje vaz1i i u neshto opshtijem sluchaju, za $\mathcal{L}(X;Y)$, ukoliko je $X$ Banahov, a $Y$ normirani vektorski prostor.
\end{nap}
\\ \\
\begin{tma}
(Teorema o konachnom prirashtaju) Neka su $X$ i $Y$ normirani vektorski prostori, $V \in X$ otvoren i $f:V \to Y$ neprekidno. Ako je 
\[ [a, a+h] = \{a+th \mid 0 \leq t \leq 1\} \psj V \]
i ako je preslikavanje $f$ diferencijabilno u svim tachkama skupa 
\[]a, a+h[ = \{a+th \mid 0 < t < 1\}\]
onda vaz1i 
\[\norm{f(a+h) - f(a)} \leq \sup_{x \in ]a, a+h[} \norm{f'(x)} \cdot \norm{h}\]
\end{tma}
\begin{proof}
Dokazati da teorema vaz1i na svakom segmentu $[c_1, c_2] \psj ]a, a+h[$. Pretpostaviti suprotno, 
\[ \sup_{x \in [c_1, c_2]} \norm{f'(x)} < \frac{\norm{\Delta f}}{\norm{\Delta x}} \]
Podeliti interval $[c_1, c_2]$ na dva podintervala i primeniti pomoc1nu lemu na njih. Pro\-du\-z1i\-ti postupak, formirati niz intervala. Ponovo primeniti pomoc1nu lemu. Izvesti kon\-tra\-dik\-ci\-ju.
\end{proof}

\begin{nap}
Pomoc1na lema iz prethodne teoreme je sledec1a nejednakost: \\ \\
Ako za $\alpha, \beta, \gamma, a, b, c \in \R_+ $ vaz1i $c \leq a+b$ i $\gamma = \alpha + \beta$, onda je
\[\frac{c}{\gamma} \leq \max \{ \frac{a}{\alpha}, \frac{b}{\beta} \} \]
\end{nap}
\\ \\
\begin{tma}
(Ojlerova teorema) Neka je $X$ normirani vektorski prostor nad $\R$. Di\-fe\-ren\-ci\-ja\-bil\-na funkcija $f: X \to \R$ je homogena stepena $k>0$ ako i samo ako je \[df(x) \cdot x = k f(x)\]
\end{tma}
\begin{proof}
Definishati pomoc1nu funkciju $\psi (t) = f(tx) - t^kf(x)$ i odatle dokazati oba smera ekvivalencije. 
\end{proof}

\begin{tvr}
$\mathcal{L} (X; \mathcal{L}(X;Y)) \cong \mathcal{L} (X,X; Y)$
\end{tvr}
\begin{proof}
Neka je $L \in \mathcal{L} (X; \mathcal{L}(X;Y))$ proizvoljno. Definisati $\tilde{L} : X \times X \to Y$ kao \[\tilde{L} (\eta, \xi) := L(\eta)(\xi)\]
Dokazati da je sa $\varphi(L) = \tilde{L}$ dobro definisan traz1eni izomorfizam.
\end{proof}

\begin{tvr}
	$fD^n a \implies D^nf(a)$ je simetrichno $n$-linearno preslikavanje.
\end{tvr}
\begin{proof}
	Primetiti najpre da je dovoljno dokazati za transpozicije, jer one generishu grupu permutacija.
	Uvesti \[F_{\zeta, \eta} (t) := f(a +t(\zeta + \eta)) - f(a + t\zeta) - f(a + t\eta) - f(a)\] i 
	dokazati da je \[ D^2f(a)(\zeta, \eta) = \lim_{t\to 0} \dfrac{F_{\zeta, \eta} (t)}{t^2} .\] 
	Izvesti zakljuchak iz toga, chinjenice da je $F$ simetrichno po $\zeta$ i $\eta$, kao i 
	\[ D^nf(a) (\zeta_1, \zeta_2, \dotso, \zeta_n) = D_{\zeta_1}D_{\zeta_2}\dotso D_{\zeta_n} f(a).\]
\end{proof}

\begin{tma}
Neka su $X, Y$ normirani vektorski prostori, $V \psj X$ otvoren, $a \in V$, $f \mathcal{D}^{n-1}V$ i $f\mathcal{D}^n a$. Tada je 
\[f(a+h) = f(a) + f'(a)h + \dots + \frac{1}{n!} f^{(n)}(a) h^n + o(\norm{h}^n), h \to 0\]
\end{tma}
\begin{proof}
Dokaz izvesti indukcijom po $n$ korish\-c1e\-njem Posledice 2. teoreme o konachnom prirashtaju i chinjenice da je $(Lx^n)' = n Lx^{n-1}$.
\end{proof}

\subsection{Unutrashnje tachke ekstremuma}

\begin{tvr}
	Neka je $V$ otvoren podskup normiranog vektorskog prostora $X$, $a \in V$, $f:V\to \R$, $f D^n a$,
	$Df(a) = D^2f(a) = \dotso = D^{n-1}f(a)$, $D^n f(a) \neq 0 $. Da bi $a$ bila tachka lokalnog ekstremuma
    funkcije $f$:
	\begin{itemize}
		\item \textbf{neophodno} je da $n$ bude paran broj i da je $D^nf(a)$ semidefinitna forma.
		\item \textbf{dovoljno} je da su vrednosti $D^nf(a)h^n$ odvojene od $0$ na jedinichnoj sveri $\norm h = 1$.
	\end{itemize}
\end{tvr}
\begin{proof}
	Primeniti Tejlorovu formulu.
\end{proof}


\begin{pr}(Ojler--Lagranzhove jednachine)
	%TODO
\end{pr}

\subsection{Teorema o implicitnoj funkciji}

\begin{tma}
(Teorema o implicitnoj funkciji) Neka su $X, Y$ i $Z$ normirani vektorski prostori, pri chemu je $Y$ kompletan, $x_0 \in X$, $y_0 \in Y$ i $W = B]x_0; \alpha[ \times B]y_0; \beta[ \subseteq X \times Y$. Pretpostavimo da preslikavanje $F:W \to Z$ ispunjava sledec1e uslove:
\begin{enumerate}
\item[(1)] $F(x_0, y_0) = 0$
\item[(2)] $FC(x_0, y_0)$
\item[(3)] $D_2F$ definisano na $W$ i neprekidno u tachki $(x_0, y_0)$
\item[(4)] $\exists {(D_2F(x_0, y_0))}^{-1} \in \mathcal{L} (Z; Y)$
\end{enumerate}
Tada postoje okoline $U$ i $ V$ tachaka $x_0$ i $y_0$ i preslikavanje $f: U \to Y$ sa svojstvima:
\begin{enumerate}
\item[(a)] $U \times V \subseteq W$
\item[(b)] $[(x,y)\in U \times V$ i $F(x,y) = 0] \iff y = f(x)$
\item[(v)] $fCx_0$

\end{enumerate}
\end{tma}
\begin{proof}
Poshto su translacije homeomorfizmi, bez umanjenja opshtosti pretpostaviti da vaz1i $x_0 = 0 $ i $y_0 = 0$. Definisati pomoc1nu funkciju $g_x: B]0; \beta[ \to Y$ kao \[g_x(y) := y - {(D_2F(0, 0))}^{-1} F(x,y).\] Koristec1i Teoremu o konachnom prirashtaju dokazati da je $g_x$ kontrakcija i da slika neki kompletan skup u sebe. Primeniti Banahov stav o fiksnoj tachki. Odatle izvesti zakljuchak. 
\end{proof}
\begin{tma}
(Teorema o inverznoj funkciji) Neka su $X, Y$ normirani vektorski prostori, pri chemu je $Y$ kompletan, neka je $V \subseteq Y$ otvoren skup, $y_0 \in V$ i $g: V \to X$ preslikavanje koje ima sledec1a svojstva:
\begin{enumerate}
\item[(1)] $g \in \mathcal{D}(V;X)$
\item[(2)] $DgCy_0$
\item[(3)] $\exists {(Dg(y_0))}^{-1} \in \mathcal{L} (X; Y)$
\end{enumerate}
Tada postoje okolina $V_0 \subseteq Y$ tachke $y_0$ i okolina $U_0 \subseteq X$ tachke $x_0 := g(y_0)$ takve da je $g: V_0 \to U_0$ bijekcija, $g^{-1} \mathcal{D} x_0$ i vaz1i \[ Dg^{-1}(x_0) = {(Dg(y))}^{-1}.\]
\end{tma}
\begin{proof}
Dokaz sledi iz Teoreme o implicitnoj funkciji primenjene na funkciju $F(x,y) = x - g(y)$.
\end{proof}
\begin{tma}(Teorema o rangu) Neka je $V \subseteq \R^k$ otvoren i $f: V \to \R^l$ preslikavanje klase $C^n$, takvo da je za sve $x \in V$ $\rang Df(x) = r$. Tada u okolini svake tachke $x_0 \in V$ i njene slike $y_0 = f(x_0) \in \R^l$ postoje lokalne koordinate (difeomorfizmi) klase $C^n$ u kojima $f$ ima zapis \[f: (s_1, \dots , s_k) \to (s_1, \dots, s_r, 0, \dots , 0).\]
Krac1e recheno, preslikavanje konstantnog ranga $r$ lokalno izgleda kao projekcija na $\R^r$.
\end{tma}
\begin{proof}
	Ideja dokaza je da \zn ispravimo\zng koordinate za prvih $r$ koordinata.
	Najpre, kako je permutacija koordinata difeomorfizam, mozhemo da pretpostavimo da je 
	glavni minor matrice $Df$ nesingularan, tj. $\det \left( \frac{\partial f_i}{\partial x_j}\right)_{1\leq i, j \leq r} \neq 0$.
	Definishimo preslikavanje $\varphi$ sa 
	\begin{align*}
		\varphi_j (x_1, \dotso, x_k):= 
		\begin{cases}
			f_j(x_1, \dotso, x_k), &1\leq j \leq r \\
			x_j, &r<j\leq k
		\end{cases}
	\end{align*}
	Ono jeste lokalni difeomorfizam, jer je $D\varphi \neq 0$ i vazhi 
	\begin{align*}
		f_j\circ \varphi^{-1}(s_1,\dotso, s_k) = 
		\begin{cases}
			s_j, &1 \leq j \leq r \\ 
			f_j(s_1, \dotso , s_k), &r < j \leq k
		\end{cases}
	\end{align*}
	Sada zakljuchiti kako izgleda matrica prvog izvoda $D(f\circ \varphi^{-1})$ i zbog chinjenice da je $\varphi$ lokalni
	difeomorfizam zakljuchiti da je ona istog ranga kao i $Df$, kao i da je $\frac{\partial f_i}{\partial x_j}= 0,r< i, j \leq k$,
	odakle sledi da $f_j(s_1, \dotso , s_k)$ zavisi samo od $s_1,\dotso, s_r$. Na kraju, definisati $\psi$ sa 
	\begin{align*}
		\psi_j (y):= 
		\begin{cases}
			y_j, &1\leq j \leq r \\
			y_j - f_j(y_1,\dotso, y_r), &r<j\leq l
		\end{cases}
	\end{align*}
	Zakljuchiti da je $\psi$ lokalni izomorfizam i da je $\psi\circ f\circ \varphi^{-1}$ trazhena kompozicija.

\end{proof}
\begin{tvr}
Svaki difeomorfizam klase $C^1$ 
\[\R^l \supseteq V \xrightarrow[]{f} f(V) \subseteq \R^l \]
moz1e lokalno da se predstavi kao kompozicija $l$ prostih difeomorfizama. 
\end{tvr}
\begin{proof}
Indukcijom po $k$ dokazati da difeomorfizam koji menja najvishe $k$ koordinata moz1e lokalno da se predstavi kao kompozicija $k$ prostih difeomorfizama.
\end{proof}

\section{Mnogostrukosti}

\subsection{Podmnogostrukosti u $\R^n$ i uslovni ekstremumi}

\begin{tvr}
Za $M \psj \R^{k+l}$ sledec1a tvrd1enja su ekvivalenta:
\begin{itemize}
\item[(a)] M je $k$ - dimenziona podmnogostrukost klase $C^p$
\item[(b)] $(\forall p \in M)$ postoje otvorene okoline $p \in V$ i $0 \in U$ u $\R^{k+l}$ i difeomorfizam $g: U \to V$ klase $C^p$ takav da je $g(V \cap M) = U \cap (\R^k \times {0})$
\item[(v)] $(\forall p \in M)$ postoje otvorene okoline $p \in V$  u $\R^{k+l}$ i $0 \in U$ u $\R^k$ i imerzija $h: D \to V$ klase $C^p$ takva da je $h: D \to V \cap M$ homeomorfizam u relativnoj topologiji na $V \cap M$ nasled1enoj iz $R^{k+l}$.
\end{itemize}
\end{tvr}
\begin{proof}
Kako je $k$-dim podmnogostrukost u $\R^{k+l}$ lokalno zadata jednachinom $f(x) = 0$, gde je $f$ sumbmerzija i $\rang Df(x) = l$, iz Teoreme o rangu zakljuchiti da je moguc1e izabrati lokalne koordinate u kojima $f$ ima zapis 
\[ f(x_1, \dots , x_{k+l}) = (x_1, \dots, x_l). \] Odatle direktno zakljuchiti (b). Uz pomoc1 prethodnog, definisati traz1enu imerziju i zakljuchiti (v).
\end{proof}

\begin{tvr}
Neka je $h:D \to M$ lokalna parametrizacija okoline tachke $p = h(0)$ i neka je $f: V \to \R^l$ submerzija takva da je $M \cap V = f^{-1} (0)$. Tada vaz1i da je $T_p M = \ker D f(p)$.
\end{tvr}
\begin{proof}
Diferenciranjem $f(h(t)) \equiv 0$ u tachki $t = 0$ i primenom pravila za izvod kom\-po\-zi\-ci\-je preslikavanja zakljuchiti da je $T_p M \psj \ker Df(p)$. Odatle, primenom Prve teoreme o i\-zo\-mor\-fi\-zmu na preslikavanje $Df(p)$ zakljuchiti da vaz1i jednakost.
\end{proof}

\begin{tma}
Neka je $V \in \R^n$ otvoren skup, $f: V \to \R$ funkcija klase $C^1$ i $M \psj V$ glatka podmnogostrukost. Da bi tachka $p \in M$ bila tachka uslovnog lokalnog ekstremuma funkcije $f|_M$ neophodno je da bude ispunjen bar 1 od sledec1ih uslova:
\begin{itemize}
\item[(a)] $df(p) = 0$ (tj. $p$ je kritichna tachka za $f$)
\item[(b)] $T_pM \psj T_pS$, gde je $S:= \{x \in V | f(x) = f(p) \}$
\end{itemize}
\end{tma}
\begin{proof}
Pretpostaviti da vaz1i $a)$ u teoremi. Odatle direktno zakljuchiti da je $p$ kandidat za bezuslovni, a samim tim i uslovni ekstremum. Dalje, pretpostaviti da $a)$ u teoremi ne vaz1i. Dokazati da je onda u nekoj okolini tachke $p$ funkcija $f$ submerzija, pa je onda $S$ $(n-1)$ - dim glatka podmnogostrukost, pa ima smisla $T_pS$. Uzeti $\gamma: ]-\delta, +\delta[ \to M$ takvo da $\gamma(0) = p$. Iz $(f \circ \gamma)'(0) = 0$ izvesti $T_pM \psj T_pS$.
\end{proof}

\begin{nap}
Neka je $M$ globalno zadato jednachinom $g(x) = 0$ za neku submerziju $g: V \to \RR$ i neka je $p$ lokalni ekstremum za $f|_M$. Sa $\gamma : I \to M$ data je neka kriva koja lez1i na $M$ i za koju vaz1i $\gamma(0) = p$. Tada, $f \circ \gamma$ ima lokalni ekstremum u $0$, tj. za $f \circ \gamma : \RR \to \RR$ vaz1i $(f \circ \gamma)'(0) = 0$ odnosno $\overrightarrow{\nabla f}(\gamma(0))\cdot \gamma'(0) = 0$ za svako $\gamma$ kroz $p$. Odatle sledi $\overrightarrow{\nabla f}(p) \perp M$. Kako je $\gamma \in M$, odatle sledi $(g \circ \gamma)(t) = 0$ pa kada to diferenciramo po $t$ dobijamo da je $\overrightarrow{\nabla g} \perp M$. Konachno, dobijamo sistem $n+1$ jednachina sa $n+1$ nepoznatih koji je odred1en i koji nam sluz1i za odred1ivanje uslovnih ekstremuma:
\begin{empheq}[box=\fbox]{align*}
\overrightarrow{\nabla f}(p) &= \lambda \overrightarrow{\nabla g}(p)\\
g(p) &= 0 
\end{empheq}
\end{nap}

\subsection{Apstraktne mnogostrukosti}

\begin{lema}
Neka je $M \subseteq \RR^{k+l}$ $k$ - dimenziona podmnogostrukost klase $C^p$, neka su $V_1, V_2$ okoline (u $\RR^{k+l}$) tachke $p \in M$ i neka su $h_1 : D_1 \to V_1$, $h_2: D_2 \to V_2$ dve lokalne $C^p$ parametrizacije skupova $V_1 \cap M$ i $V_2 \cap M$. Tada je
\[h_2 \circ {h_1}^{-1} : D_1 \cap {h_1}^{-1}(V_2) \to D_2 \cap {h_2}^{-1}(V_1)\]
difeomorfizam klase $C^p$.
\end{lema}
\begin{proof}
Dokaz leme sledi iz chinjenice da je kompozicija difeomorfizama difeomor-fizam.
%\begin{center}
%\includegraphics[width=5cm]
%{Two_coordinate_charts_on_a_manifold.png}
%\end{center}
\end{proof}

\begin{tvr}
Ne postoji imerzija klase $C^p$ gde je $p \leq 1$ gde $h: \RR \to \RR^2$ takva da je $h(\RR) = \Gamma$.
\end{tvr}
\begin{proof}
Iz neprekidnosti $h'$ izvesti kontradikciju.
\end{proof}

\begin{tma}
Za svako otvoreno pokrivanje ${\{U_{\lambda} \}}_{\lambda \in \Lambda}$ glatke mnogostrukosti $M$ postoji razla-ganje jedinice ${\{\rho_{\lambda} \}}_{\lambda \in \Lambda}$ takvo da je $\latin{\textrm{supp}} \rho_{\lambda} \subseteq U_{\lambda}$.
\end{tma}
\\ \\
\begin{tma}
Za ${\{U_{\lambda} \}}_{\lambda \in \Lambda}$ kao u prethodnoj teoremi postoji razlaganje jedinice ${\{\rho_{\sigma} \}}_{\sigma \in \Sigma}$ sa kompaktnim nosachem, takvo da $(\forall \sigma \in \Sigma)(\exists \lambda \in \Lambda)$ $\latin{\textrm{supp}} \rho_{\sigma} \subseteq U_{\lambda}$.
\end{tma}
\\ \\
\begin{tma}
(Vitnijeva) Svaka glatka $n$-dimenziona mnogostrukost M moz1e da se glatko uloz1i u $\RR^{2n+1}$, tj. postoji $C^{\infty}$ ulaganje  $f:M \to \RR^{2n+1}$. Drugim rechima, svaka apstraktna glatka mnogostrukost je difeomorfna podmnogostrukosti u euklidskom prostoru.
\end{tma}
\\ \\
\begin{tma}
Svaka kompaktna glatka mnogostrukost $M$ moz1e glatko da se uloz1i u $\RR^l$ za dovoljno veliko $l$.
\end{tma}
\begin{proof}
Neka je $\{{(U_j, \varphi_j)}_{1 \leq j \leq k} \}$ atlas na $M$, takav da je $\varphi_j(U_j) = B(0;3)$ i da $V_j = {\varphi_j}^{-1} (B(0;1))$ takod1e pokrivaju M. Za svako $j$ konstruisati funkciju $h_j : U_j \to \RR$ pomoc1u 
\[h(x) = \frac{g(4 - \norm{x}^2)}{g(4 - \norm{x}^2) + g(\norm{x}^2 - 1)}\]
gde je $g$ dato sa
\[
g(t)=
\begin{cases}
e^{-\frac{1}{t}}, \hspace{0.5cm} t>0\\
0, \hspace{0.5cm} t\leq0
\end{cases}
\]
Definisati preslikavanje \[f:M \to \underbrace{\RR^n \times \dots \times \RR^n}_{k}\times \underbrace{\RR \times \dots \times \RR}_{k} =: \RR^l\]
sa 
\[f(p) = \biggl(h_1(p)\varphi_1(p), \dots , h_k(p)\varphi_k(p), h_1(p), \dots , h_k(p)\biggr)\]
gde je $n = \dim M$. Dokazati da je $f$ jedno ulaganje.
\\
\end{proof}

\subsection{Izvod preslikavanja $f:M \to N$}

\begin{lema}
	Derivacija $D$ na $C^\infty(p)$ preslikavanja koje je konstantno u nekoj okolini tachke $p$ je 0.
\end{lema}
\begin{proof}
	Primeniti Lajbnicovo pravilo i iz homogenosti (derivacija je linearno pres-likavanje, dakle i homogeno) zakljuchiti
	da je dovoljno dokazati $D1 = 0$.
\end{proof}

\begin{lema}
	Neka je $f \in C^\infty_\RR (U)$ i $p \in U$. Tada postoje lopta $B := B(p, \eps) \psj U$ i funkcije
	$g_1, g_2, \dotso, g_n \in C^\infty_\RR (U)$ takve da:
	\begin{enumerate}
		\item $g_j(p) = \dfrac{\partial f}{\partial x_j}(p)$
		\item $\ds{f(x_1, x_2, \dotso, x_n) = f(p) + \sum_{j=1}^n(x_j - p_j)g_j(x)}$
	\end{enumerate}
\end{lema}
\begin{proof}
	Za $x \in B$ ($B$ kao u postavci) mozhemo napisati
	\begin{align*}
		f(x) &= f(p) + \int_0^1 \frac{1}{dt}f(p+t(x-p))dt\\
			 &= f(p) + \sum_{j=1}^n(x_j - p_j)\int_0^1 \frac{\partial f}{\partial x_j}f(p+t(x-p))dt
	\end{align*}
	Uzeti da je \(\ds{ g_j = \int_0^1 \frac{\partial f}{\partial x_j}f(p+t(x-p))dt }\).
\end{proof}

\begin{tma}
	Neka je $U\psj \RR^n$ otvoren skup i $p\in U$. Za svaku derivaciju $D : C^\infty(p)\to C^\infty(p)$
	algebre $C^\infty(p)$ postoji jedinstveni vektor $\xi \in T_p U$ takav da je $Df(p) = df(p)\cdot \xi$.
\end{tma}
\begin{proof}
	Napisati $f(x)$ kao u prethodnoj lemi, i primeniti derivaciju na taj zapis. Dobi-jemo
	\[ Df(x) = \sum_{j=1}^n(D\pi_j(x)\cdot g_j(x) + (x_j - p_j)\cdot Dg_j(x)) ,\] 
	Primenom ovoga u tachki $p$ imamo:
	\[ Df(p) = \sum_{j=1}^nD\pi_j(p)\cdot \dfrac{\partial f}{\partial x_j}(p) = df(p)\cdot \xi ,\]
	za $\xi = (D\pi_1, D\pi_2, \dotso, D\pi_n)$.
\end{proof}

\begin{nap}
	Prethodna teorema nam daje ekvivalenciju geometrijskog i algebarskog (analitichkog) pristupa
	definiciji tangnetnog prostora mnogostrukosti.
\end{nap}

\begin{posl}
	$T_p M$ je vektorski prostor, jer je linearna kombinacija derivacija takodje derivacija.
\end{posl}

\begin{posl}
	$\dim M = n \implies \dim T_p M = n$
\end{posl}
\begin{proof}
	Neka je $\varphi : U \to \RR^n$ lokalna karta i $p \in U$. Tada je 
	$\dfrac{\partial }{\partial x_1} \Big|_p, \dfrac{\partial }{\partial x_2}\Big|_p, \dotso, \dfrac{\partial }{\partial x_n}\Big|_p$
	baza za $T_{\varphi(p)} U$, pa je 
	$(\varphi^{-1})_*\dfrac{\partial }{\partial x_1}\Big|_p, (\varphi^{-1})_*\dfrac{\partial }{\partial x_2}\Big|_p, \dotso, (\varphi^{-1})_*\dfrac{\partial }{\partial x_n}\Big|_p$
	baza za $T_p M$.
\end{proof}

\section{Integracija}

\subsection{Apstraktni integrali}

\begin{lema}
	Vektorski potprostor $\mathcal{E}$ prostora $\RR^T$ je reshetka akko vazhi
	\[ f \in \mathcal{E} \implies |f| \in \mathcal{E} .\]
\end{lema}
\begin{proof}
	Iskoristiti vezu izmedju apsolutne vrednosti i maksimuma i mimimuma.
\end{proof}

\begin{tvr}
	Radonov integral je ujedno i elementarni integral.
\end{tvr}
\begin{proof}
	Po definiciji je radonov integral linearan, tako da je dovoljno dokazati da 
	za opadajuc1i niz funkcija $f_n$ sa kompaktnim nosachem koji tezhi $0$
	vazhi i da je $I(f_n) \to 0$, kad $n\to \infty$. Neka je $K_n$ nosach funkcije
	$f_n$. Kako je $f_n$ opadajuc1i niz koji tezhi nuli, sledi da je $K_1\supseteq K_2 \supseteq \dotso$.
	Iz Dinijeve teoreme ($0$ je neprekidna funkcija) sledi da $f_n\rightrightarrows 0$, pa za proizvoljno
	$\eps > 0$ postoji $n_0 \in \NN$, takvo da $n\geq n_0 \implies f_n\mid_{K_1} < \eps$.

	Na kraju, neka je $\psi \in  C_C(T)$ funkcija za koju je $0 \leq \psi \leq 1$ i $\psi \mid_{K_1} \equiv 1$ (takva
	funkcija postoji zbog Urisonove leme). Zbog $\mathrm{supp} f_n \psj K_1$, vazhi
	\[ f_n = f_n \cdot \chi_{K_1} \leq f_n \cdot \psi \leq \eps \psi .\] 
	Iz pozitivnosti i linearnosti Radonovog integrala imamo $I(f_n) \leq \eps I(\psi)$, pa je onda i $I(f_n)\to 0$, kad
	$n\to \infty$.
\end{proof}

\begin{nap}
	Prostor $C_C(T)$ je prostor neprekidnih funkcija $f:T\to \RR$ koje imaju kompaktan nosach, pri chemu je 
	$T$ lokalno kompaktan metrichki (ili Hausdorfov) prostor.
\end{nap}

\begin{pr}
	Josh neki primeri elementarnog integrala su:
	\begin{itemize}
		\item Dirakova delta funkcija $\delta_a: C_C(T)\to \RR$, $\delta_a(f):=f(a)$ (valuacija u tachki).
		\item Rimanov integral
		\item Riman-Stiltjesov integral
	\end{itemize}
\end{pr}

\begin{lema}
	Za Radonov integral $I: C_C(T) \to \RR$ vazhi:
	\begin{enumerate}
		\item $f \leq g \implies I(f) \leq I(g)$
		\item $|I(f)| \leq I(|f|)$
	\end{enumerate}
\end{lema}
\begin{proof}
	Za prvi deo tvrdjenja primeniti linearnost i pozitivnost $I$. Za drugi deo primeniti prvi deo i $-|f| \leq f \leq |f|$.
\end{proof}

\begin{tvr}
	Neka je $T$ lokalno kompaktan Hausdorfov prostor, $K \psj T$ kompaktan podskup i $C(T; K)$ skup funkcija i $C_C(T)$
	kojima je nosach u $K$. Tada je restrikcija
	\[ I:C(T;K) \to \RR \] 
	Radonovog integrala neprekidna u odnosu na normu
	\[ \norm{f}_K := \sup\{|f(t)| \mid t\in K\} .\] 
\end{tvr}
\begin{proof}
	Neka je $\psi \in  C_C(T)$ funkcija za koju je $0 \leq \psi \leq 1$ i $\psi \mid_{K} \equiv 1$. Tada je 
	\[ (\forall f \in C(T; K))(\forall t \in T) |f(t)| \leq \norm{f}_K \cdot \psi(t) ,\] 
	pa je $|I(f)| \leq I(|f|) \leq I(\psi) \cdot \norm{f}_K$.
\end{proof}

\begin{tma}(Fubini)
	Neka su $S$ i $T$ komapktni Hausdorfovi prostori i neka su 
	\[ I_S : C(S) \to \RR ,  I_T : C(T) \to \RR\] 
	Radonovi integrali. Tada postoji jedinstven Radonov integral 
	\[ I:C(S\times T) \to \RR \] 
	takav da je 
	\[ I(f\otimes g) = I(f)\cdot I(g) .\] 
\end{tma}

\begin{nap}
	Tenzorski proizvod funkcija $f:S\to \RR$ i $g:T\to \RR$ je funkcija $f\otimes g:S\times T \to \RR$
	definisana sa 
	\[ f\otimes g(s, t) := f(s)\cdot g(t) \] 
	Definishemo i prostor $C(S)\otimes C(T)$ kao
	\[ C(S)\otimes C(T) := \Big\{\sum_j f_j\otimes g_j\mid f_j \in C(S), g_j \in C(T)\Big\},\] 
	gde je navedena suma konachna.
\end{nap}
\begin{proof}
	Iz prethodnog tvrdjenja sledi da je Radonov integral na kompaktnim prostorima neprekidan,
	pa je i ravnomerno neprekidan po Kantorovoj teoremi. Iz Ston-Vajershtrasove teoreme sledi
	da je $\overline{C(S)\otimes C(T)} = C(S\times T)$, pa dokaz tvrdjenja sledi iz Principa
	produzhenja ravnomerno neprekidnog preslikavanja sa kompletnim kodomenom (odatle 
	sledi i jedinstvenost).
\end{proof}

\begin{posl}(Fubinijeva teorema za Rimanov integral)
	Neka je $\mathcal{J} = \mathcal{J}_1 \times \mathcal{J}_2$ $n$-dimenzioni kvadar,
	predstavljen kao proizvod $k$--dimenzionog kvadra $\mathcal J_1$ i $(n-k)$--dimenzionog kvadra $\mathcal J_2$.
	Tada, za neprekidnu funkciju $f:\mathcal{J}\to \RR$ vazhi:
	\[ \int_{\mathcal J} = \int_{\mathcal J_2} \left(\int_{\mathcal J_1}f(x, y) dx \right) dy = \int_{\mathcal J_1} \left(\int_{\mathcal J_2}f(x, y) dy \right) dx.\] 
\end{posl}
\begin{proof}
	Zbog linearnosti Rimanovog integrala i prethodne teoreme, dovoljno je dokazati tvrdjenje
	za $f = f_1 \otimes f_2$, za $f_i:\mathcal J_i \to \RR, i \in \{1, 2\}$ neprekidne funkcije.
	Sada direktno sledi dokaz iz definicije Rimanovog integrala po kvadru $\mathcal J$.
\end{proof}

\subsection{Normirani prostor integrabilnih funkcija}

\begin{lema}
	Relacija ekvivalencije na skupu 
	$\mathcal D (T,\mathcal{E}, I) := \{f \in D(T,\mathcal E, I) \mid \overline{I}(|f|)< +\infty\}$,
	definisana sa 
	\[ f \sim g \iff f = g \text{ skoro svuda na T} .\] 
	je kongruencija (tj. saglasna je sa operacijama vektorskog prostora). Kolichnichki prostor je 
	normirani vektorski prostor, sa normom
	\[ \norm{[f]}_1 := \overline{I}(|f|) ,\] 
	koja je dobro definisana.
\end{lema}
\begin{proof}
	Navedena norma je ochigledno pseudo--norma (zadovoljava nejednakost trougla i simetrichnost). Njena dobra definisanost 
	je jasna, ako je $[f] = [g]$, onda je $f = g$ skoro svuda na $T$, pa je i $\overline{I}(|f|) = \overline{I}(|g|)$.
	Treba josh dokazati da je $[\alpha f + \beta g] = \alpha [f] + \beta [g]$. Homogenost je jasna, pa je dovoljno dokazati
	samo linearnost, koja c1e sledeti iz chinjenice da je unija dva skupa mere nula takodje skup mere nula.
\end{proof}
\begin{posl}
	$(\widetilde{D} (T,\mathcal{E}, I), \norm{\cdot}_1)$ je normirani vektorski prostor.
\end{posl}

\begin{nap}
	Preslikavanje $I:\widetilde{D} (T,\mathcal{E}, I) \to \RR$ je neprekidno linearno preslikavanje,
	jer je $|I(f)| \leq I(|f|) =:\norm{f}_1$.
\end{nap}

\subsection{Kompletiranje metrichkog prostora i Lebegov integral}

\begin{tma}
	Za svaki metrichki prostor $(M_0, d_0)$ postoji kompletan metrichki prostor
	$(M,d)$ i in\-jektivno preslikavanje $i:M_0 \ to M$ sa svojstvima:
	\begin{enumerate}
		\item $d(i(x), i(y)) = d_0(x, y), \forall x, y \in M_0$
		\item $i(M_0)$ je gust u $M$.
	\end{enumerate}
\end{tma}
\begin{proof}
	Na prostoru $X\psj M_0^\NN$, svih Koshijevih nizova u $M_0$, definisati relaciju ekviva\-lencije
	\[ \{a_n\} \sim \{b_n\} \iff \lim_{n\to \infty} d_0(a_n, b_n) = 0 .\] 
	Na prostoru $M:= X/_\sim$ definishemo metriku $\ds{d([a_n], [b_n]):= \lim_{n\to \infty} (a_n, b_n)}$.

	Dobra definisanost metrike $d$ sledi iz aditivnosti po limesu, a lako je i dokazati da $d$ jeste metrika.
	Definishimo ulaganje $i:M_0 \to M$ sa $i(x) := [\{x\}]$, gde je $\{x\}$ konstantan niz. Posmatrajmo $x \in M$.
	Tada je $x = [x_n]$, za neki niz $x_n$ u $M_0$ i vazhi $i(x_n) \to x$, odakle sledi da je $i(M_0)$ gust u $M$.
	Treba josh dokazati da je $M$ kompletan. Neka je $x_n$ Koshijev niz u $M$. Iz $\overline{i(M_0)} = M$ sledi 
	da postoji $a_n \in M_0$ za koje je $d(i(a_n), x_n) < \frac{1}{n}$. Zakljuchujemo da ako je $x_n$ Koshijev,
	onda je i $i(a_n)$ Koshijev (kao niz konstantnih nizova u $M$). Medjutim, odatle sledi i da je $a_n$ Koshijev
	u $M_0$, pa $a_n$ definishe tachku u $M$ i vazhi $x_n \to a$.
\end{proof}

\begin{nap}
	Ovo se mozhe i primeniti da od normiranog vektorskog prostora dobijemo Banahov prostor. Dokaz 
	se mozhe izvesti elegantnije, zbog chinjenice da je dual uvek kompletan i zbog chinjenice 
	da postoji linearno ulaganje normiranog vektorskog prostora u dual duala.
\end{nap}

\begin{posl}
	Kompletiranje prostora $(C(\mathcal J), \norm{\cdot}_1)$ naziva se prostorom funkcija inte\-grabilnih po Lebegu. 
	Kako je integral ravnomerno neprekidan, onda se on mozhe (iz Principa produzhenja ravnomerno neprekidne funkcije
	sa kompletnim kodomenom) pro\-duzhiti do preslikavanja koje nazivamo Lebegovim integralo.
\end{posl}

\subsection{Klasa funkcija integrabilnih po Rimanu}

\begin{lema}
	\( f \in \mathcal R (\mathcal J) \implies f \text{ je ogranichena na } \mathcal J \) 
\end{lema}
\begin{proof}
	Pretpostaviti suprotno, i zakljuchitida za svaku podelu postoji kvadar na kojem je funkcija
	neogranichena, tj. postoje tachke $\zeta, \eta$ iz tog kvadra takve da je \( |f(\zeta) - f(\eta)| \) 
	proizvoljno veliko. Uzeti dve podele sa uochenim tachkama koje se razlikuju samo na tom kvadru,
	jednu sa uochenom tachkom $\zeta$, a drugu sa $\eta$. Izvesti kontradikciju iz Koshijevog kriterijuma
	postojanja limesa po filteru.
\end{proof}

\begin{pr}
	Grafik neprekidne funkcije $f: E \to \RR$, $E \psj \RR^{k-1}$ je skup Lebegove mere $0$.
\end{pr}
\begin{proof}
	Najpre, tvrdjenje se mozhe dokazati iz ravnomerne neprekidnosti u sluchaju da je $E$ zatvoren kvadar.
	Da bismo ovo proshirili na proizvoljan skup $E\psj \RR^{k-1}$, dovoljno je da pokrijemo grafik funkcije
	sa prebrojivo mnogo skupova mere $0$. Posmatrajmo skup $S:= \QQ^{k-1}\cap E$. On je gust u $E$ i prebrojiv,
	pa je dovoljno dokazati da je u nekoj okolini svake tachke iz $S$, grafik restrikcije
	na tu okolinu skup mere $0$. To je ochigledno, jer oko svake tachke skupa $S$ mozhemo posmatrati zatvoren
	kvadar koji je sadrzhan u $E$, a na tom kvadru je grafik restrikcije mere $0$, zbog razmatranja na pochetku 
	zadatka. Prebrojiva unija skupova mere $0$ je skup mere $0$, pa je i grafik cele funkcije $f$ mere $0$.
\end{proof}

\begin{pr}
	Neka je $0 < l < k$ i $\pi: \RR^k\to \RR^l$ projekcija $\RR^k$ na neki $l-$dimenzioni potprostor.
	Ako je $E\psj \RR^k$ skup takav da je $\pi(E)$ skup mere nula u $\RR^l$, onda je i $E$ skup mere $0$
	u $\RR^k$.
\end{pr}
\begin{proof}
	Prethodni primer mozhemo dokazati i kada je kodomen $\RR^n$. Lako je videti da je \( E\psj \pi(E)\times \RR^{k-l} \).
	Primenimo prethodni zadatak na konstantnu funkciju, odakle sledi da je $\pi(E)\times \RR^{k-l}$ skup mere $0$.
	Tada je i $E$ skup mere $0$, kao podskup skupa mere $0$.
\end{proof}

\begin{tma}(Darbuova)
	Za svaku ogranichenu funkciju $f:\mathcal J \to \RR$ vazhi:
	\begin{align*}
		\overline{\int_{\mathcal J}}f(x)dx = \lim_{\f _{\mathcal J}} S(f, \mathcal P)\\
		\underline{\int_{\mathcal J}}f(x)dx = \lim_{\f _{\mathcal J}} s(f, \mathcal P)
	\end{align*}
\end{tma}
\begin{proof}
	Dovoljno je dokazati jedno od navedenih tvrdjenja (drugo sledi primenom na $-f$). Oznachimo $\underline I = \underline{\int_{\mathcal J}}$
	Neka je $\mathcal P_\eps$ podela takva da je $s(f, \mathcal P _\eps) > \underline I - \eps$. Posmatrati $\Gamma_\eps$, skup 
	svih tachaka koje lezhe na granici nekog od kvadara podele $\mathcal P_\eps$ i dokazati da je on Lebegove mere 0.
	Onda postoji $\lambda_\eps > 0$ takvo da je za svaku podelu chiji je parametar manji od $\lambda_\eps$, zbir zapremina
	kvadara koji seku $\Gamma_\eps$ manji od $\eps$. Neka je $\mathcal P$ takva podela i neka je $\mathcal P'$ podela koja
	sadrzhi sve tachke podela $\mathcal P$ i $\mathcal P_\eps$. Tada je 
	\[ \underline I - \eps < s(f, \mathcal P_\eps) < s(f, \mathcal P') \leq \underline I .\] 
	Zakljuchiti da u razlici $s(f, \mathcal P') - s(f,\mathcal P)$ uchestvuju samo sabirci koji odgovaraju kvadrima koji
	seku $\Gamma_\eps$, pa je $|s(f, \mathcal P') - s(f,\mathcal P)| < 2M\eps$, gde je $|f| \leq M$. Na kraju dokazati da 
	je $\underline I = s(f, P) < (2M+1)\eps$, odakle sledi tvrdjenje.
\end{proof}

\begin{posl}(Darbuov kriterijum integrabilnosti po Rimanu)
	\[ f \in \mathcal R(\mathcal J) \iff f \text{ je ogranichena i } \underline \int f(x)dx = \overline \int f(x)dx\] 
\end{posl}

\begin{lema}
	Neka je $T\psj \RR^n$ i $f:T \to\RR$ ogranichena funkcija i $\eps > 0$. Tada je skup
	$$F:=\{t \in T \mid \omega(f; T) \geq \eps\}$$
	zatovren u $T$
\end{lema}
\begin{proof}
	Dokazati da je komplement $T\setminus F$ otvoren.
\end{proof}

\begin{tma}(Lebegov kriterijum integrabilnosti po Rimanu)
	Funkcija $f: \mathcal J \to \RR$ je integrabilna po Rimanu akko je ogranichena i neprekidna skoro svuda u smilsu Lebega 
	(tj. ako je skup prekida \( E = \{x \in \mathcal J \mid \omega(f; x) > 0\} \) mere 0.)
\end{tma}
\begin{proof}
	($\implies$)
	Najpre, zakljuchiti da je dovoljno dokazati 
	\[ (\forall n \in \NN) E_n :=  \left\{x \in \mathcal J \mid \omega(f; x) > \frac{1}{n}\right\} \text{ je mere 0.}\] 
	Neka je $\mathcal P$ podela takva da je $S(f, \mathcal P) - s(f, \mathcal P) < \frac{\eps}{n}$ i neka je 
	\( P_n := \{\mathcal J \in \mathcal P \mid \mathcal J \cap E_n \neq \emptyset\} \).
	Odatle izvesti zakljuchak da je \( \ds{\sum_{I \in P_n} \mu(I) < \eps} \), pa je $E_n$ mere 0 ($P_n$ pokrivaju $E_n$).

	($\impliedby$)
	Posmatrati $E_\eps := \{ x\in \mathcal J \mid \omega(f; x) \geq \frac{\eps}{2\mu(J)}\}$. Kao podskup skupa $E$,
	ovaj skup je mere nula, a zbog prethodne leme je i zatvoren, pa c1e biti i kompaktan. Onda postoji konachna familija 
	kvadara koja ga pokriva i za koju je $\mu(J_1) + \mu(J_2) + \dotso + \mu(J_k) < \frac{\eps}{4\norm{f}_\infty}$.
	Za $t \in \mathcal J \setminus \bigcup_{i=1}^k J_k$ vazhi \( \omega(f; t) < \frac{\eps}{2\mu(J)}\), tj.
	za svako takvo $t$ postoji kvadar $I_t$ takav da je $\sup_{I_t}f - \inf_{I_t}f< \frac{\eps}{2\mu(J)} $.
	Kako je skup $t \in \mathcal J \setminus \bigcup_{i=1}^k J_k$ zatovren, mozhemo izdvojiti konachno potpokrivanje $I_t$.
	Na kraju, razliku $S(f, \mathcal P) - s(f, \mathcal P)$ predstaviti kao $\sum (M_i - m_i) \mu(K_i)$ i razdvojiti je 
	na sumiranje po kvadrima $J_i$ i $I_t$, odakle c1emo dobiti $S(f, \mathcal P) - s(f, \mathcal P) < \eps$.
\end{proof}

\begin{posl}
	Neka je $E\psj \RR^K$ ogranichen skup sadrzhan u kvadru $\mathcal J$. Tada je 
	\[ \chi_E \in \mathcal R (\mathcal J) \iff \partial E \text{ je skup mere 0} .\] 
\end{posl}
\begin{proof}
	$\partial E$ je skup prekida funkcije $\chi_E$.
\end{proof}

\begin{tvr}
	Neka je $\mathcal J \psj \RR^k$ $k-$dimenzioni kvadar. Tada je prostor $(C(\mathcal J), \norm{\cdot}_1)$
	gust u $(\mathcal R(\mathcal J), \norm{\cdot}_1)$.
\end{tvr}
\begin{proof}
	Pokriti skup prekida prebrojivom unijom kvadara $\{J_n\}_{n \in \NN}$, chiji je zbir zapre\-mina manji od $ \frac{\eps}{2M}$. Ideja
	je da \textit{neprekidno} predjemo sa kvadra $J_n$ na ostatak prostora, a to c1emo uraditi konstruisanjem funkcije
	$\varphi_n$ koja je jednaka $0$ na $J_n$, varira izmedju $0$ i $1$ na kvadru $\widetilde{J}_n$ koji je dvaput vec1e zapremine
	od $J_n$ i sadrzhi ga, i jedanaka je $1$ van $\widetilde{J}_n$. Kombinovanjem svih tih funkcija dobijamo funkciju $g$,
	za koju vazhi 
	\[  \norm{f - g}_1 = \int_{\bigcup \widetilde{J}_n} |f-g| < \eps.\] 
\end{proof}

\begin{tvr}
	Fubinijeva teorema vazhi za klasu funkcija integrabilnih po Rimanu.
\end{tvr}
\begin{proof}
	Sledi iz ranije dokazane teoreme za neprekidne funkcije i principa produzhenja neprekidnog preslikavanja
	$\int_{I\times J}: C(I\times J) \to \RR$ na skup $\mathcal R(I\times J)$, u kome je $C(I\times J)$ gust.
\end{proof}

\begin{nap}
	Pre smo Fubinija dokazali za Rimanov integral neprekidne funkcije, sada smo dokazali za Rimanov integral funkcije 
	integrabilne po Rimanu.
\end{nap}

\begin{tma}
	Neka je $\mathcal J \psj \RR^k$ $k-$dim kvadar i $E\psj \mathcal J$ skup merljiv po Zhordanu. Predstavimo $\mathcal J$
	kao proizvod $\mathcal J = \mathcal J_x \times \mathcal J_y$ kvadara $\mathcal J_x \psj \RR^l$ i $\mathcal J_y \psj \RR^{k-l}$.
	Tada je za svako $y_0 \in \mathcal J_y$ presek $E_{y_0} := \{(x, y) \in E \mid y = y_0\}$ skupa $E$ i $l-$dimenzione ravni $y=y_0$
	merljiv po Zhordanu i vazhi 
	\[ \mu_k(E) = \int_{\mathcal J_y} \mu_l (E_y)dy ,\] 
	gde je $\mu_k$ (odnosno $\mu_l$) Zhordanova mera u $\RR^k$ (odnosno $\RR^l$)
\end{tma}
\begin{proof}
	Sledi iz Fubinijeve teoreme primenjene na funkciji $\chi_E(x, y) = \chi_{E_y}(x)$
\end{proof}

\subsection{Smena promenljive u integralu}

\begin{tma}
Neka su $D_t$ i $D_x$ organicheni otvoreni skupovi u $\RR^k$, $\varphi: D_t \to D_x$ difeo\-morfizam (klase $C^1$), $\varphi ' : D_t \to \RR^{k \times k}$ njegova matrica prvog izvoda i $f \in \mathcal R (D_x)$ funkcija chiji je nosach $\latin{\textrm{supp}} f := \{\overline{x \in D_x \mid f(x) \ne 0 }\}$ kompaktan u $D_x$. Tada vaz1i:
\begin{itemize}
\item[(1)] $(f \circ \varphi) |\det \varphi ' | \in \mathcal R(D_t)$
\item[(2)] $\int_{D_x} f(x) dx = \int_{D_t} f \circ \varphi(t) |\det \varphi ' (t)| dt$
\end{itemize}
\end{tma}
\begin{nap}
Dokaz1imo prvo $4$ pomoc1ne leme.
\end{nap}
\\ \\
\begin{lema}
Neka su $D_t, D_x, \varphi$ kao u teoremi, $E_t \subseteq D_t$, $E_x := \varphi(E_t) \subseteq D_x$. Tada vaz1i:
\begin{itemize}
\item[(1)] Ako je $E_t$ skup mere nula u smislu Lebega, onda je takav i $E_x$.
\item[(2)] Ako je Z1ordanova mera skupova $E_t$ i $\overline{E_t}$ nula, onda je i Z1ordanova mera skupova $E_x$ i $\overline{E_x}$ nula.
\item[(3)] Ako je skup $E_t$ merljiv po Z1ordanu i $\overline{E_t} \subseteq D_t$ onda je i skup $E_x$ merljiv po Z1ordanu i $\overline{E_x} \subseteq D_x$.
\end{itemize}
\end{lema}
\begin{proof}
Primetiti da se svaki otvoren skup $D \in \RR^k$ moz1e predstaviti kao prebrojiva unija kvadara chije se unutrashnjosti med1usobno ne seku. Kako je prebrojiva unija skupova mere nula skup mere nula, dokazati $(1)$ pod pretpostavkom da je $E_t \subseteq J$ za neko $J$. Neka je $\{ I_n \}_{n \in \NN}$ pokrivanje skupa $E_t$ kvadrima takvim da je $\sum \mu (I_j) < \epsilon$. Primenom Teoreme o konachnom prirashtaju i prethodnog zakljuchiti da se svaki skup $\varphi (I_j) $ nalazi u kvadru $\tilde{I}_j $ takvom da je $\mu (\tilde{I}_j) = M^k \mu (I_j)$. Odatle izvesti $(1)$. 

Deo $(2)$ sledi iz $(1)$ jer je $\overline{E}_t$, a samim tim i njegova neprekidna slika $\overline{E}_x$, kompaktan skup, a svaki kompaktan skup mere nula u Lebegovom smislu ima Z1ordanovu meru nula.

Deo $(3)$ izvesti iz chinjenice da difeomorfizam slika unutrashnje tachke u unutrashnje tachke, shto je posledica Teoreme o inverznoj funkciji, odnosno izvesti zakljuchak iz $\partial E_x = \varphi (\partial E_t)$.
\end{proof}

\begin{posl}
Pod uslovima Teoreme o smeni promenljive, integral u tachki $(2)$ postoji.
\end{posl}
\begin{proof}
Sledi iz Lebegove teoreme o integrabilnosti i prethodne leme.
\end{proof}

\begin{lema} (jednodimenzioni sluchaj) Neka je $\varphi: I_t \to I_x$ difeomorfizam intervala $I_t \subseteq \RR$ na interval $I_x \subseteq \RR$ i neka $f \in \mathcal R (I_x)$ i vaz1i 
\[\int_{I_x} f(x) dx = \int_{I_t} f \circ \varphi(t) | \varphi ' (t)| dt\]
\end{lema}

\begin{proof}
Dokazati primenom Njutn - Lajbnicove formule iz Analize 1.
\end{proof}

\begin{nap}
Kod smene promenljive za funkcije jedne promenljive u uobichajenom obliku nemamo apsolutnu vrednost kod $\varphi '$. To nije u koliziji, zbog toga shto se u u toj apsolutnoj vrednosti samo krije znak koji "obrc1e" granice intervala.
\end{nap}
\\ \\
\begin{nap}
Pojava $|\cdot|$ sugerishe vezu integracije i orijentacije. Ne moz1emo integraliti po objektima koji nisu orijentabilni. No, ipak postoji metoda da
se zaobidje ova prepreka, uz pomoc1 chinjenice da svaku neorijentabilna mnogostrukost ima dvolisno orijentabilno natkrivanje, po kojem onda
mozhemo vrshiti integraciju.
\end{nap}
\\ \\
\begin{lema}
Teorema o smeni promenljive u integralu vaz1i za proste difeomorfizme.
\end{lema}
\begin{proof}
Ova lema je u sushtini posledica prethodne leme i Fubinijeve teoreme. Pret\-po\-sta\-vi\-ti radi jednostavnosti zapisa da $\varphi$ menja samo poslednju ($k$ - tu) koordinatu. Kvadre $J_x \supseteq D_x$ i $J_t \supseteq D_t$ u $\RR^k$ predstaviti kao Dekartov proizvod kvadra u $\RR^{k-1}$ i intervala. Primeniti Fubinijevu teoremu na integral funkcije $f \chi_{D_x}$ po kvadru $J_x$ imajuc1i u vidu da je $\varphi$ prost difeomorfizam te je $\det \varphi ' = \frac{\partial \varphi_k}{\partial t_k}$.
\end{proof}

\begin{lema}
Ako Teorema o smeni promenljive vaz1i za difeomorfizme 
\[\psi: D_s \to D_t, \hspace{0.6cm} \varphi : D_t \to D_x\]
onda vaz1i i za kompoziciju $\varphi \circ \psi : D_s \to D_x$.
\end{lema}
\begin{proof}
Dokaz izvesti iz $\det(\varphi \circ \psi)' = \det \varphi ' \cdot \det \psi '$ i dve primene Teorema o smeni promenljive: prvo na funkciju $f$ i di\-fe\-o\-mor\-fi\-zam $\varphi$, a zatim na $f \circ \varphi | \det \varphi '|$ i di\-fe\-o\-mor\-fi\-zam $\psi$.
\end{proof}
\noindent
Moz1emo sada dovrshiti dokaz prve teoreme.

\begin{proof}(Dokaz teoreme o smeni promenljive u integralu)
Dokaz zavrshiti korish\-c1enjem Tvrd1enja o predstavljanju difeomorfizma kao kompozicije prostih, med1utim obratiti paz1nju na \textit{lokalno} u njemu. Ideja da se dokaz1e da teorema vaz1i na celom $D_t$ jeste da se $D_t$ podeli na manje skupove na kojima teorema vaz1i.
Ovde c1e biti neophodno primeniti Lebegovu lemu o kompaktnosti na skup \( \mathrm{supp} (f\circ \varphi \cdot |\det \varphi'|) \).
\end{proof}

\begin{nap}
Intuicija za malo drugachiji dokaz:
Neka je $J$ kvadar, $f: J \to \RR$ neprekidna funkcija i $\varphi: J \to \varphi(J)$ difeomorfizam klase $C^1$. Posmatrajmo shta je $\int_{\varphi(J)} f(x) dx$. Podelimo kvadar $J$ na kvadre $J_l$. Tada je \[\int_{\varphi(J)} f(x) dx = \sum_l \int_{\varphi(J_l)} f(x) dx\]
Dalje imamo da je
\[\min_{\varphi(J_l)} f \int_{\varphi(J_l)} 1 dx \leq \int_{\varphi(J_l)} f(x) dx \leq \max_{\varphi(J_l)} f \int_{\varphi(J_l)} 1 dx\]
odnosno
\[\min_{\varphi(J_l)} f \leq \frac{1}{\mu(\varphi(J_l))}\int_{\varphi(J_l)} f(x) dx \leq \max_{\varphi(J_l)} f \]
Iz neprekidnosti $f$, primenom teoreme o srednjoj vrednosti zakljuchiti da postoji $\xi_l$ takvo da
\[f(\xi_l) =\frac{1}{\mu(\varphi(J_l))}\int_{\varphi(J_l)} f(x) dx\]
Sledi da je \[\int_{\varphi(J)} f(x) dx = \sum_l f(\xi_l) \mu(\varphi(J_l))\]
 Uzmimo $\tau_l \in J_l$ takvo da $\xi_l = \varphi(\tau_l)$. Tada je
\[\int_{\varphi(J)} f(x) dx = \sum_l f \circ \varphi(\tau_l) \mu(\varphi(J_l))\]
Dokazati josh da je $\mu(\varphi(J_l)) = |\det \varphi| \cdot \mu(J_l)$. U sluchaju da je $\varphi$ linearno, dokaz izvesti koristec1i teoremu o jedinstvenosti determinante. U sluchaju da $\varphi$ nije linearno pred-staviti ga kao $\varphi \approx C + \varphi '$, gde je $\varphi '$ linearno a $C = \varphi(x_0)$ translacija, stoga ne menja zapreminu.
\end{nap}

\section{Vektorska polja i diferencijalne forme}

\begin{tvr}
Za $\alpha \in \Omega^k(M)$, $\beta \in \Omega^l(M)$ i glatko preslikavanje $F: M \to N$ vaz1i:
\begin{itemize}
\item[(1)] $F^* (\alpha + \beta) = F^* \alpha + F^* \beta$
\item[(2)] $F^* (\alpha \wedge \beta) = F^* \alpha \wedge F^* \beta$
\end{itemize}
Drugim rechima, ako definishemo \[\Omega^*(M) = \bigoplus^{\dim M}_{j=0} \Omega^j (M)\] i slichno $\Omega^*(N) $ onda je 
\[F^*: (\Omega^*(N), +, \wedge) \to (\Omega^*(M), +, \wedge)\]
homomorfizam algebri.
\end{tvr}

\begin{proof}
\begin{itemize}
\item[(1)] $F^*(\alpha + \beta) X = (\alpha + \beta) (F_* X) = \alpha(F_* X) + \beta(F_* X) = (F^* \alpha + F^* \beta) X$
\item[(2)]
\begin{align*}
 F^* (\alpha \wedge \beta) (X,Y) &= (\alpha \wedge \beta) (F_* (X), F_* (Y)) \\
 &= \begin{vmatrix} \alpha(F_*(X)) & \alpha(F_*(Y)) \\ \beta(F_*(X)) & \beta(F_*(Y)) \end{vmatrix} \\
 &= \alpha(F_*(X))\cdot \beta(F_*(Y)) - \alpha(F_*(Y)) \cdot \beta(F_*(X)) \\
 &= F^*(\alpha) X \cdot F^*(\beta) Y - F^*(\alpha) Y \cdot F^*(\beta) X \\
 &=  \begin{vmatrix} F^*(\alpha)X & F^*(\alpha)Y \\ F^*(\beta)X & F^*(\beta)Y \end{vmatrix} \\
 &= (F^*(\alpha) \wedge F^*(\beta))(X,Y)
\end{align*}
\end{itemize}
\end{proof}

\begin{nap} \latin{Pullback} \fontencoding{OT2}\selectfont se takod1e slaz1e i sa spoljashnjim diferencijalom, odnosno vaz1i \[F^* \circ d  = d \circ F^* \]
za $F: U \to V$ preslikavanje klase $C^{\infty}$, gde su $U \in \RR^n$ i $V \in \RR^k$.
 Zahvaljujuc1i ovome, moguc1e je definisati $d: \Omega^l(M) \to \Omega^{l+1}(M)$ za proizvoljnu mno\-go\-stru\-kost. 
\end{nap}
\\ \\

\begin{tvr} (Lajbnicovo pravilo)
Neka su $\alpha \in \Omega^l(M)$ i $\beta \in \Omega^k(M)$ dve forme. Tada vaz1i 
\[d(\alpha \wedge \beta) = d \alpha \wedge \beta + (-1)^l \alpha \wedge d \beta \]
\end{tvr}
\begin{proof}
Zbog linearnosti $d$ dovoljno je dokazati formulu za $\alpha = f(x) dx_{j_1} \wedge \dots \wedge dx_{j_l}$ i $\beta = g(x) dx_{i_1} \wedge \dots \wedge dx_{i_k}$. Primeniti definiciju $d$ i izvesti zakljuchak.
\end{proof}

\begin{tma} 
$\alpha \in \Omega^l(M) \implies d(d \alpha) = 0$
\end{tma}

\begin{proof}
Dva puta primeniti definiciju $d$ na $f dx_{j_1}\wedge \dots \wedge dx_{j_l} $ i iskoristiti anti\-ko\-mu\-ta\-tiv\-nost $\wedge$.
\end{proof}

\begin{posl}
$\mathrm{Im} \hspace{0.1cm} d \psj \mathrm{Ker} \hspace{0.1cm} d$
\end{posl}
\\ \\
\begin{tvr}
Ako je $F: M \to N$ difeomorfizam, onda je $F* : \hdr{l} (N) \to \hdr{l} (M)$ izomorfizam.
\end{tvr}

\begin{proof}
Iskoristiti chinjenicu da je ${(G \circ F)}^* = F^* \circ G^*$. Odatle zakljuchiti ${(F^{-1})}^* = {(F^*)}^{-1}$.
\end{proof}

\begin{tvr}
Za svako neprekidno preslikavanje $f: M \to N$ glatkih mnogostrukosti postoji glatko preslikavanje $F:M \to N$ takvo da je $f \simeq F$.
\end{tvr}

\begin{proof}
Izvesti iz Vajershtrasove teoreme o aproksimaciji i razlaganja jedinice. 
%Kako? pomoc
\end{proof}

\begin{tvr}
(Kartanova formula) $\phi_t : M \to M$ glatka (po $t$) familija glatkih pre\-sli\-ka\-va\-nja. Neka je $X(\phi_t(p)) := \frac{d \phi_t}{dt} (p)$ (pishemo krac1e: $X$). Za $\alpha \in \Omega^l(M)$ vaz1i:
\[\frac{d}{dt} {\phi_t}^* \alpha = {\phi_t}^*(X \lrcorner d \alpha + d(X \lrcorner \alpha))\]
\end{tvr}
\begin{proof}
Indukcijom po $l$.
\end{proof}

\begin{posl}
Neka su $f_0, f_1 : M \to N$, ${f_0}^*, {f_1}^* : \hdr{*}(N)\to \hdr{*}(M)$. Tada vaz1i 
\[f_0 \simeq f_1 \implies f_0^* = f_1^*\]
\end{posl}

\begin{posl}
Ako je $M \simeq N$ onda je $\hdr{*}(M) \cong \hdr{*}(N)$.
\end{posl}
\\ \\
\begin{nap}
Mnogostrukost $M$ je povezana ako  i samo ako je $\hdr{0} (M) = \RR$. 
\end{nap}
\\ \\
\begin{nap}
Opshtije: $\hdr{0} (M) = \RR^m$, gde je $m$ broj komponenti povezanosti $M$.
\end{nap}
\\ \\
\begin{nap}
Ako $l \in \{0,n\}$ onda vaz1i $\hdr{l} (\mathbb{S}^n) = \RR$, inache $\hdr{l} (\mathbb{S}^n) = 0$.
\end{nap}
\\ \\
\begin{tvr}
Ne postoji glatko preslikavanje $f: \mathbb{B}^n \to \mathbb{S}^{n-1} := \partial \mathbb{B}^n$ takvo da je $f(x) = x$ za sve $x \in \mathbb{S}^n$. Odnosno, ne postoji glatka retrakcija $\mathbb{B}^n$ na $\partial \mathbb{B}^n$.
\end{tvr}
\begin{proof}
Definisati $j: \mathbb{S}^{n-1} \to \mathbb{B}^n$ kao $j(x) = x$. Pretpostaviti suprotno, da postoji traz1ena retrakcija $f: \mathbb{B}^n \to \mathbb{S}^{n-1}$, td. $\forall a \in \mathbb{S}^{n-1}$ $f(a) = a$. Posmatrati $f \circ j$ i $\hdr{n-1}$ posmatranih objekata ($\mathbb{S}^{n-1}$ i $\mathbb{B}^n$) i izvesti konstradikciju.
\end{proof}
\begin{nap}
Ne postoji neprekidna retrakcija mnogostrukosti na granicu.
% jel ovo validno? i zasto neprekidna, nije li retrakcija po def nepr
\end{nap}
\\ \\
\begin{tma}
(Brauerova teorema o fiksnoj tachki) \\ Svako neprekidno preslikavanje $f: \mathbb{B}^n \to \mathbb{B}^n$ ima bar jednu fiksnu tachku. Drugim rechima
\[f \in C(\mathbb{B}^n, \mathbb{B}^n) \implies (\exists x_0 \in \mathbb{B}^n) f(x_0) = x_0 \]
\end{tma}
\begin{proof}
Pretpostaviti suprotno, da $\forall x \in \mathbb{B}^n$ vaz1i $f(x) \ne x$. Definisati $r(x) = \frac{f(x) - x}{\norm{f(x) - x}}$, $r: \mathbb{B}^n \to \mathbb{S}^{n-1}$. Pokazati da je $r$ retrakcija. Odatle, na osnovu prethodnog, izvesti kontradikciju.
\end{proof}

\section{Dualni pogled na diferencijalne forme}

\begin{tvr}
	\( \div \circ \rot = 0, \rot \circ \grad = 0 .\)
\end{tvr}
\begin{proof}
	$\div (\rot \vec B) = \vec \nabla \cdot (\vec \nabla \times \vec B) = 0$, $\rot (\grad f) = \vec \nabla \times (\vec \nabla  f)$.
\end{proof}

\begin{nap}
	Vektor $\vec \nabla$ mozhe da se s\-hvati kao analogija diferencijala $d$, a prethodno tvrdjenje kao $d\circ d = 0$.
	Uspostaviti izomorfizam izmedju diferencijalnih formi i odgovarajuc1ih prostora glatkih funkcija na kojima definishemo
	$\grad, \rot, \div$ i dokazati da c1e svi dijagrami komutirati.
\end{nap}

\begin{tvr}
	Svaka $1-$forma u $\RR^n$ je oblika $\omega^1_{\vec B} := \langle \vec B, \cdot \rangle$, za jedinstveno $\vec B$ i 
	svaka $2-$forma u $\RR^3$ je oblika $\omega^2_{\vec B} := (\vec B, \cdot, \cdot) $ za jedinstveno $\vec B$.
\end{tvr}
\begin{proof}
	Sledi iz chinjenice da su to potprostori celog prostora iste dimenzije kao i ceo prostor.
\end{proof}

\begin{nap}
	Prethodno tvrdjenje opravdava sledec1u definiciju:
	\begin{align*}
		d\omega^0_f =:\omega^1_{\grad f} \\
		d\omega^1_{\vec B} =:\omega^2_{\rot \vec B} \\
		d\omega^2_{\vec B} =:\omega^3_{\div \vec B}
	\end{align*}
	Iz ove definicije je jasno da \( \div \circ \rot = 0, \rot \circ \grad = 0 \) sledi iz $d \circ d = 0$.
\end{nap}

\begin{pr}
	\( \omega^1_{\vec A}\wedge \omega^1_{\vec B} = \omega^2_{\vec A \times \vec B}\),
	\( \omega^1_{\vec A}\wedge \omega^2_{\vec B} = \omega^3_{\vec A \cdot \vec B}\) 
\end{pr}
\begin{proof}
	Napisati $ \omega^1_{\vec A}$ kao $A_1dx + A_2dy + A_3dz$ i slichno sa ostalim formama i dokazati u 
	koordinatnom zapisu.
\end{proof}


\section{Furijeova analiza}

\begin{lema}
Neka je $\{e_1, \dots, e_n\}$ ortonormiran sistem i $f$ proizvoljan vektor. Tada je vektor 
\[h = f - \sum_{j=1}^n \langle f, e_j \rangle e_j\]
ortogonalan na prostor $\mathcal{L}(\{e_1, \dots, e_n\})$.
\end{lema}
\begin{proof}
Dokazati da za svako $l$ vaz1i $\langle h, e_l \rangle = 0$ i iz linearnosti zakljuchiti traz1eno.
\end{proof}

\begin{posl}
Svaki vektor $f$ moz1e da se napishe kao 
\[f = f_e + h\]
gde je $f_e \in \mathcal{L}(\{e_1, \dots, e_n \})$ i $h$ ortogonalno na $\mathcal{L}(\{e_1, \dots, e_n \})$.
\end{posl}
\\ \\
\begin{lema}
(Pitagorina teorema) Neka je $g\bot h$ i $f = g+h$. Tada je \[\norm{f}^2 = \norm{g}^2 + \norm{h}^2\]
\end{lema}
\begin{proof}
Raspisati $\norm{f}^2 = \langle f, f \rangle = \langle g+h, g+h \rangle $.
\end{proof}

\begin{posl}
Ako je $f_1, \dots, f_n$ ortogonalan sistem i $f = f_1 + \dots + f_n$, onda je 
\[\norm{f}^2 = \norm{f_1}^2 + \dots + \norm{f_n}^2 \]
\end{posl}
\\ \\
\begin{posl}
Ako je $e_1, \dots , e_n$ ortonormiran sistem i $f = c_1 e_1 + \dots + c_n e_n$, onda je 
\[\norm{f}^2 = {|c_1|}^2 + \dots + {|c_n|}^2\]
\end{posl}
\\ \\
\begin{tvr}
Neka je $e_1, \dots , e_n$ ortonormiran sistem i $f$ proizvoljan vektor. Tada je \[f_e := \sum_{j=1}^n \langle f, e_j \rangle e_j\] najbliz1i vektor u $\mathcal{L}(\{e_1, \dots, e_n \})$ vektoru $f$.
\end{tvr}
\begin{proof}
Primenom Pitagorine teoreme dokazati da vaz1i $\norm{f-g}^2 \geq \norm{f - f_e}^2$
\end{proof}

\begin{tvr}
(Beselova nejednakost) Ako je $e_1, \dots, e_n$ ortonormirani sistem, onda za svaki vektor $f$ vaz1i
\[\sum_{j=1}^n {| \langle f, e_j \rangle |}^2 \leq \norm{f}^2 \]
\end{tvr}
\begin{proof}
Zapisati $f$ kao $f = f_e + h$, gde je $f_e\bot h$ i $f_e = \sum_{j=1}^n \langle f, e_j \rangle e_j$. Primeniti pitagorinu teoremu.
\end{proof}

\begin{tma}
Ako je $\{e_1, e_2, \dots \}$ ortonormiran sistem, sledec1i iskazi su ekvivalentni:
\begin{itemize}
\item[a)] sistem $\{e_1, e_2, \dots \}$ je potpun
\item[b)] za svaki vektor $f$ vaz1i
\[f = \sum_{n=1}^{\infty} \langle f, e_n \rangle e_n\]
\item[v)] za svaki vektor $f$ vaz1i \textit{Parsevalova jednakost}:
\[\norm{f}^2 = \sum_{n=1}^{\infty} {|\langle f, e_n \rangle |}^2 \]
\end{itemize}
\end{tma}
\begin{proof}
Trivijalno iz prethodnih tvrd1enja.
\end{proof}

\begin{posl}
Ortonormirani sistem vektora je potpun ako i samo ako je baza.
\end{posl}
\begin{proof}
$\boxed{\Leftarrow}:$ vaz1i uvek \\
$\boxed{\Rightarrow}:$ je a) $\Longrightarrow$ b) u prethodnoj teoremi
\end{proof}

\begin{pr}
Sistem $\{1,x,x^2, \dots \}$ je potpun u pred - Hilbertovom prostoru $C_{\CC}[-1,1]$ sa skalarnim proizvodom $\int_{-1}^1 f(x) g(x) dx$, ali on nije baza!
\end{pr}
\\ \\
\begin{tma}
Neka je $X$ Hilbertov prostor, $\{e_1, e_2 \dots \}$ ortonormiran sistem i $f \in X$ proizvoljan vektor. Tada vaz1i:
\begin{itemize}
\item[(1)] Furijeov red $\sum_{n=1}^{\infty} \langle f, e_n \rangle e_n$ konvergira ka nekom vektoru $f_e \in X$.
\item[(2)] $f = f_e + h$, gde je $h \bot \mathcal{L}(\{e_1, \dots, e_n \})$
\end{itemize}
\end{tma}
\begin{proof}
\begin{itemize}
\item[(1)] Na osnovu Beselove nejednakosti $\sum {|\langle f, e_n \rangle |}^2$ konvergira. Na osnovu Pita\-go\-ri\-ne teoreme zakljuchiti da $\sum \langle f, e_n \rangle e_n $ zadovoljava Koshijev kriterijum, pa zbog kom\-ple\-tno\-sti prostora i konvergira.
\item[(2)] Dokazati da za svako $j$ vaz1i $\langle h, e_j \rangle = 0$
\end{itemize}
\end{proof}

\begin{tma}
Neka je $X$ pred-Hilbertov prostor i $\{f_1, f_2, \dots\}$ sistem linearno nezavisnih vektora.
\begin{itemize}
\item[(1)] Da bi ovaj sistem bio potpun neophodno je da u $X$ ne postoji vektor koji je razlichit od $0$ i ortogonalan na sve $f_j$.
\item[(2)] Ako je prostor $X$ Hilbertov, uslov iz $(1)$ je i dovoljan za potpunost sistema.
\end{itemize}
\end{tma}
\begin{proof}
\begin{itemize}
\item[(1)] Pretpostaviti suprotno i na osnovu Pitagorine teoreme izvesti kon\-tra\-dik\-ci\-ju sa potpunosh\-c1u sistema.
\item[(2)] Gram-SHmitovim postupkom ortonormirati sistem. Na osnovu prethodne teoreme zakljuchiti da se svaki vektor $f \in X$ moz1e zapisati kao $f = f_e + h$, gde je $f_e = \sum \langle f, e_n \rangle e_n$ i $h \perp \mathcal{L}(\{e_1, \dots, e_n\}$. Sledi da je $h = 0$. Izvesti krajnji zakljuchak iz prethodnog.
\end{itemize}
\end{proof}

\begin{pr}
Neka je $f:[-1,1] \to \CC$ neprekidna funkcija, takva da je $\forall n$ 
\[\int_{-1}^1 x^n f(x) dx = 0\]
Tada je na osnovu prethodne i Prve Vajershtrasove teoreme, $f\equiv 0$.
\end{pr}
\\ \\
\begin{pr}
Sistem $\{1, x, x^2, \dots \}$ je linearno nezavisan i potpun u pred-Hilbertovom pro\-sto\-ru $C_{\CC}[-1,1]$ sa skalarnim proizvodom $\int_{-1}^1 f(x)\overline{g(x)} = \langle f, g \rangle$ ali nije baza u tom prostoru! \\ \ Pretpostaviti suprotno, neka jeste baza. Tada se svako $f \in C_{\CC}[-1,1]$ moz1e zapisati kao red $f(x) = \sum c_n x^n$. \\ Obratiti paz1nju da se konvergencija tog reda odnosi na konvergenciju u normi iznduko\-va\-noj skalarnim proizvodom zadatim ranije. Kako opshti chlan konvergentnog reda tez1i nuli, zakljuchiti da $\norm{c_n x^n} \to 0$. Dalje, iz definicije skalarnog proizvoda zakljuchiti da je $\norm{c_n x^n} = |c_n| \sqrt{\frac{2}{2n+1}}$ i da zbog $\norm{c_n x^n} \to 0$, za dovoljno veliko $n$ vaz1i $|c_n| < \sqrt{2n+1}$. \\ Koristec1i ovu nejednakost dokazati da stepeni red $\sum c_n x^n$ konvergira na intervalu $]-1, 1[$. Odnosno, tada je funkcija $g: ]-1, 1[ \to \CC$, $g(x) = \sum_{n=0}^{\infty} c_n x^n$ dobro definisana funkcija. Poshto posmatramo konvergencije u razlichitim normama, mora se paz1ljivo pristupati detalju na kraju ovog dokaza. Nije ochigledno da je $f = g$. To se moz1e pokazati imajuc1i u vidu sledec1e: $\int_{-1 + \epsilon}^{1 - \epsilon} {|f-g|}^2 \leq {\norm{f - \sum c_n x^n}}^2 \to 0$. Odatle sledi da je $f \in C^{\infty}]-1, 1[$, med1utim to nije tachno, te dobijamo kontradikciju.
\end{pr}
\\ \\
\begin{tma}
Svaka funkcija moz1e proizvoljno dobro da se aproksimira u normi $\norm{\cdot}_2$:
\begin{itemize}
\item[(a)] funkcijom koja je ogranichena na $]- \pi, \pi[$ i integrabilna na $[-\pi, \pi]$
\item[(b)] deo po deo konstantnom funkcijom
\item[(v)] neprekidnom funkcijom
\item[(g)] trigonometrijskom polinomom
\end{itemize}
\end{tma}
\begin{proof}
\begin{itemize}
\item[(a)] Tvrd1enje je netrivijalno samo ako se radi o nesvojstvenom integralu. Pret\-po\-sta\-vi\-ti da je nesvojstven u granichnim tachkama (sluchaj unutrashnjih razmatra se analogno).
\item[(b)] Integral ogranichene integrabilne funkcije moz1e se proizvoljno dobro aproksi\-mi\-ra\-ti donjom Darbuovom sumom, te ovo sledi iz prethodnog.
\item[(v)] Dovoljno je dokazati da konstantne funkcije iz prethodnog mogu dovoljno dobro da se aprok\-si\-mi\-ra\-ju neprekidnim funkcijama. 
\item[(g)] Dovoljno je dokazati da se neprekidna funkcija $h$ iz prethodnog moz1e proizvoljno dobro aproksimirati trigonometrijskim polinomom. Biramo $h_1$ tako da je nepre\-ki\-dna i $h_1(x) = h(x)$ za $x \in [-\pi, \pi - \delta]$ i $h_1(-\pi) = h_1(\pi)$. Odatle izvesti zakljuchak na osnovu Druge Vajershtrasove teoreme.
\end{itemize}
\end{proof}
\begin{posl}
Neka je $PC([-\pi, \pi]; \CC)$ pred-Hilbertov prostor deo-po-deo neprekidnih funkcija. Tada vaz1i:
\begin{itemize}
\item[(a)] Trigonometrijski Furijeov red proizvoljne funkcije $f \in PC([-\pi, \pi]; \CC)$ konvergira ka $f$ u normi izvedenoj iz skalarnog proizvoda \[\langle f, g \rangle = \int_{-\pi}^{\pi} f(x) \overline{g(x)} dx \]
\item[(b)] Ako trigonometrijski red
\[\sum_{n = -\infty}^{\infty} c_n e^{inx}\]
konvergira ka funkciji $f$ u normi izvedenoj iz skalarnog proizvoda, onda je on njen Furijeov red.
\item[(v)] Ako funkcije $f,g \in PC([-\pi, \pi]; \CC)$ imaju isti Furijeov red, onda je $f \equiv g$.
\end{itemize}
\end{posl}
\begin{proof}
\begin{itemize}
\item[(a)] Posledica chinjenice da je svaki potpun ortonormiran sistem baza i chinjeni\-ce da su parcijalne sume Furijeovog reda svakog vektora linearne kombinacije or\-to\-nor\-mi\-ra\-nog sistema koje najbolje aproksimiraju taj vektor.
\item[(b)] Sledi iz opshte chinjenice o jedinstvenosti razlaganja proizvoljnog vektora $f$ pred-Hilbertovog prostora po ortonormiranoj bazi.
\item[(v)] Iz prethodnog.
\end{itemize}
\end{proof}
\begin{lema}(Riman - Lebegova) Neka je $f: ]\alpha, \beta[ \to \CC$ lokalno antegrabilna i neka je apsolutno integrabilna u obichnom Rimanovom ili nesvojstvenom smislu. Tada je 
\[\lim_{\RR \ni \lambda \to \infty} \int_{\alpha}^{\beta} f(x) e^{i \lambda x} dx = 0\]
\end{lema}
\begin{proof}
Prvo, zakljuchiti da je dovoljno dokazati tvrd1enje za realnu funkciju $f$. Dalje, zakljuchiti da je dovoljno da se tvrd1enje dokaz1e za funkciju $f: [a,b] \to \RR$, odnosno da moz1emo iskljuchiti tachke $\alpha$ i  $\beta$ iz domena. Tako moz1emo postupiti sa svim singularnim tachkama. Posmatrati donju Darbuovu sumu $\sum m_j \Delta x_j$ i  funkciju $g(x) = m_j$ za $x_{j-1} \leq x \leq x_j$. Zakljuchiti da je dovoljno dokazati tvrd1enje za $g$ umesto $f$. Za $g$ dati integral se eksplicitno izrachinava.
\end{proof}
\begin{tma}
Neka funckija $f:]-\pi, \pi[ \to \CC$ zadovoljava uslove Riman-Lebegove leme i neka je $0 < \delta < \pi$. Tada za svako $x \in ]-\pi, \pi[$ vaz1i
\[\lim_{n \to \infty}\big(\int_{-\pi}^{-\delta}f(x-t)D_n(t) dt + \int_\delta^\pi f(x-t)D_n(t)dt\big) = 0\]
\end{tma}
\begin{proof}
Konstruisati pomoc1nu funkciju $g$ sa
\[
g(t) = \begin{cases}
  f(x-t) {(\sin \frac{t}{2})}^{-1}  & \delta \leq |t| \leq \pi \\
  0 & |t|>\pi
\end{cases}
\]
Izvesti zakljuchak iz chinjence da je $D_n(x) = \frac{\sin(n + \frac{1}{2})x}{\sin \frac{x}{2}}$ i Riman-Lebegove leme.
\end{proof}
\begin{posl}(Princip lokalizacije) Neka funkcije $f,g:]-\pi, \pi[ \to \CC$ zadovoljavaju uslove Riman-Lebegove leme i neka u nekoj okolini $V$ tachke $x_0 \in ]-\pi, \pi[$ vaz1i $f|_V \equiv g|_V$. Tada Furijeovi redovi 
\[f \sim \sum_{n = - \infty}^{+ \infty} c_n e^{inx_0}\]
i
\[g \sim \sum_{n = -\infty}^{+ \infty} d_n e^{inx_0} \]
funkcija $f$ i $g$ ili oba divergiraju, ili oba konvergiraju ka istoj vrednosti.
\end{posl}
\begin{proof}
Parcijalna suma Furijeovog reda funkcije $f$ u tachki $x_0$ je
\[S_n(x_0) = \frac{1}{2\pi} \int_{-\pi}^{\pi}f(x_0 - t) D_n(t) dt\]
To lako dobijamo iz chinjenice da za konachne sume integral i suma komutiraju, kao i chinjenice da je $D_n$ $2 \pi$-periodichno. Dalje, na osnovu perthodne teoreme zakljuchujemo da $\lim_{n\to \infty} S_n(x_0)$ zavisi samo od restrikcije $f|_V = g|_V$.
\end{proof}
\begin{tma}(Dinijev dovoljan uslov konvergencije Furijeovog reda)
Neka je $f: \RR \to \CC$ $2 \pi$-periodichna funkcija koja zadovoljava uslove Riman-Lebegove leme. Ako $f$ zadovoljava Dinijev uslov u tachki $x$, onda njen Furijeov red konvergira u toj tachki i vaz1i
\[\sum_{n = - \infty}^{\infty} c_n e^{inx} = \frac{f(x_-) + f(x_+)}{2}\]
\end{tma}
\begin{nap}
Ne moz1emo da ochekujemo da Furijeov red funkcije u tachki uvek konvergira ka vrednosti $f(x)$ funkcije u toj tachki.
\end{nap}
\\ \\
\begin{tma}
(Fejerova) Neka je $f: \RR \to \CC$ $2 \pi$-periodichna funkcija, apsolutno integrabilna na $[-\pi, \pi]$. Tada vaz1i:
\begin{itemize}
\item[(1)] $f \in C(\RR) \implies \sigma_n \rightrightarrows f$
\item[(2)] $fCx_0 \implies \sigma_n(x_0) \to f(x_0)$
\end{itemize}
gde je \[\sigma_n(x):= \frac{S_0(x) + \dots + S_n(x)}{n+1}\]
\end{tma}

\end{document}
